\section*{Общая характеристика работы}

\newcommand{\actuality}{\underline{\textbf{\actualityTXT}}}
\newcommand{\progress}{\underline{\textbf{\progressTXT}}}
\newcommand{\actualityandprogress}{\underline{\textbf{\actualityandprogressTXT}}}
\newcommand{\aim}{\underline{{\textbf\aimTXT}}}
\newcommand{\tasks}{\underline{\textbf{\tasksTXT}}}
\newcommand{\novelty}{\underline{\textbf{\noveltyTXT}}}
\newcommand{\influence}{\underline{\textbf{\influenceTXT}}}
\newcommand{\methods}{\underline{\textbf{\methodsTXT}}}
\newcommand{\defpositions}{\underline{\textbf{\defpositionsTXT}}}
\newcommand{\reliability}{\underline{\textbf{\reliabilityTXT}}}
\newcommand{\probation}{\underline{\textbf{\probationTXT}}}
\newcommand{\contribution}{\underline{\textbf{\contributionTXT}}}
\newcommand{\publications}{\underline{\textbf{\publicationsTXT}}}

{\actualityandprogress}. Методы восстановления измерений в задаче компьютерной томографии можно разделить на интегральные и алгебраические. К интегральным относится метод светки и обратной проекции....

{\aim} ~данной работы являются разработка метода реконструкции, позволяющего учесть присутствие в объекте сильнопоглощающих включений, а так же метода численной интерпретации результатов измерений многокомпонентных объектов.

Для достижения поставленной цели были решены следующие {\tasks}:
\begin{enumerate}
  \item построен асимптотически быстрый алгебраический метод реконструкции, основанный на применении быстрого преобразования Хафа.
  \item доказана сходимость построеного алгебраического метода реконструкции, за счет полученного математического выражения градиента быстрого преобразования Хафа.
  \item построен алгоритм реконструкции для объектов, содержащих сильнопоглощающие включения.
  \item построен алгоритм реконструкции, учитывающий покомпонентное ослабление полихроматического спектра.
\end{enumerate}

{\novelty}
\begin{enumerate}
  \item Впервые для реконструкции томографических измерений было применено быстрое преобразование Хафа.
  \item Впервые получено выражение для производной быстрого преобразования Хафа, а так же алгоритм его эффективного вычисления.
  \item Построен алгоритм реконструкции, учитывающий вклад сильно поглощающих включений с помощью оригинальной модели ограничений-неравенств.
  \item Предложена схема обработки данных полихроматического зондирования, при которой восстанавливаются реальные физические концентрации элементов.
\end{enumerate}

{\influence} ~Результаты, полученные в диссертационной работе, используются для обработки данных лабораторных исследований. Построенные алгоритмы лягут в основу программного обеспечения новых моделей промышленных томографов.

Полученное в работе выражение для градиента быстрого преобразования Хафа имеет общетеоретическое значение и уже применяется в области машинного обучения для обратного распространения ошибки в нейронных сетях глубокого обучения через слой БПХ.

{\methods}
Для решения задач реконструкции томографических измерений используются методы теории условной и безусловной оптимизации: градиентные методы оптимизации, квадратичное программирование, регуляризация.
Для ускорения итерации алгебраического метода используются алгоритмы обработки изображений в виде быстрого преобразования Хафа.


{\defpositions}
\begin{enumerate}
  \item Предложен эффективный вычислительный метод решения задачи томографической реконструкции FHT-SIRT, основанный на алгебраическом подходе, который позволяет снизить асимптотическую оценку сложности вычисления итерации с $O(n^3)$ до $O(n^2~\log n)$, что подтвержается численными экспериментами и замерами времени работы программной реализации алгоритма.
  \item Проведено математическое обоснование сходимости предложенного метода.
  \item Предложен метод реконструкции на основе квадратичного программирования, который позволяет уменьшить артефакты на восстановленном изображении, вызванные наличием сильно поглощающих областей в зондируемом объекте.
  \item Предложен алгебраический метод реконструкции для случая полихроматического зондирования, который решает оптимизационную задачу реконструкции относительно линейной комбинации концентраций с ограничениями-неравенствами на их область значений.
\end{enumerate}


{\reliability} полученных результатов обеспечивается модельными экспериментами и численными симуляциями, а так же экспериментами с восстановлением реально измеренных в лабораторных условиях образцов.\ Результаты находятся в соответствии с результатами, полученными другими авторами.


{\probation}
Основные результаты работы докладывались~на: конферециях 
35-я конференция молодых ученых и специалистов «Информационные технологии и системы» (2012, 19 - 25 августа, Петрозаводск, Россия),
11th Biennal Conference on High Resolution X-Ray Diffraction and Imaging (XTOP 2012, St. Petersburg, Russia), 
29th European Conference on Modelling and Simulation (ECMS 2015, Albena, Bulgaria),
Eighth International Conference on Machine Vision (ICMV 2015, Barcelona, Spain),
на общефизическом семинаре ИПТМ РАН (октябрь 2016).

{\contribution} Все результаты диссертации, вынесенные на защиту, получены автором самостоятельно.
Автором самостоятельно реализованы методы восстановления FHT-SIRT из первой главы, барьерных функций из второй, метод взвешанных невязок из третьей, проведены численные экспериметны по обработке реальных и модельных данных.
Постановка задач и обсуждение результатов проводились совместно с научным руководителем.
Генерация модельных данных для экспериментов в полихроматике проводилась аспирантом факультета КН НИУ ВШЭ Ингачевой А.~С. 
Программная имплементация метода мягких ограничений, использованная для сравнения с методом барьерных функций во второй главе, принадлежит Соколову В.~В.
Измерения для экспериментов по восстановлению зуба со свинцовым включением производились на лабоработном источнике ИК РАН в лаборатории рефлектометрии и малоуглового рассеяния.
Многие аспекты исследований в разное время обсуждались с Чукалиной М.~В., Николаевым Д.~П., Бузмаковым А.~В., Ингачевой А.~С., Соколовым В.~В.


\ifthenelse{\equal{\thebibliosel}{0}}{% Встроенная реализация с загрузкой файла через движок bibtex8
    \publications\ Основные результаты по теме диссертации изложены в 11 печатных изданиях, 
    3 из которых изданы в журналах, рекомендованных ВАК, 
    6 "--- в тезисах докладов.%
}{% Реализация пакетом biblatex через движок biber
%Сделана отдельная секция, чтобы не отображались в списке цитированных материалов
  \begin{refsection}%
    \printbibliography[heading=countauthorvak, env=countauthorvak, keyword=biblioauthorvak, section=1]%
    \printbibliography[heading=countauthornotvak, env=countauthornotvak, keyword=biblioauthornotvak, section=1]%
    \printbibliography[heading=countauthorconf, env=countauthorconf, keyword=biblioauthorconf, section=1]%
    \printbibliography[heading=countauthor, env=countauthor, keyword=biblioauthor, section=1]%

    \publications\ Основные результаты по теме диссертации изложены в \arabic{citeauthor} печатных изданиях \nocite{PruBuzNik13, Prun2013Crys, Vestnik2016, Prun2013AutomAndRemCont, buz_jac_2015, chukalina2014xray}, 
    \arabic{citeauthorvak} из которых изданы в журналах, рекомендованных ВАК %\nocite{PruBuzNik13, Prun2013Crys, Vestnik2016}, 
    \arabic{citeauthorconf} "--- в тезисах докладов \nocite{itas2015Prun,itas2015Ingacheva,ecms2015Chukalina, icmv2015Chukalina, embc2013Buzmakov, nikolaevfast}.
  \end{refsection}
} % Характеристика работы по структуре во введении и в автореферате не отличается (ГОСТ Р 7.0.11, пункты 5.3.1 и 9.2.1), потому её загружаем из одного и того же внешнего файла, предварительно задав форму выделения некоторым параметрам

%Диссертационная работа была выполнена при поддержке грантов ...

%\underline{\textbf{Объем и структура работы.}} Диссертация состоит из~введения, четырех глав, заключения и~приложения. Полный объем диссертации \textbf{ХХХ}~страниц текста с~\textbf{ХХ}~рисунками и~5~таблицами. Список литературы содержит \textbf{ХХX}~наименование.

%\newpage
\section*{Содержание работы}
Во \underline{\textbf{введении}} приводится общая характеристика работы, 
обоснована актуальность, цель диссертации, научная новизна, показана теоретическая и практическая значимость исследований,
проведённых в рамках данной диссертационной работы, представлены методы исследования, указана степень достоверности и апробация результатов.

В \underline{\textbf{первой главе}} формулируется математическая постановка задачи, приводится обзор актуальной научной литературе по теме.

Формулируется модель формирования измерений с параллельным пучком при зондировании монохроматическим излучением. 
Задача восстановления томографических измерений ставится как обращение преобразования Радона 
\begin{equation}\notag
  p(\varphi, \xi) = \ln \left (\frac{I_0}{I(\varphi, \xi)} \right) = 
 \iint \! \mathrm d x \mathrm d y f(x,y)\delta(x\cos\varphi + y\sin\varphi - \xi).
\end{equation}
Задача состоит в восстановлении функции ослабления рентгеновского излучения $f(x,y)$ по набору логарифмированных нормированных проекций под разными углами $p(\varphi, \xi)$.

Для решения этой задачи существует две основных группы методов: интегральные и алгебраические.


Во \underline{\textbf{второй главе}} предложен и обсуждается асимптотически быстрый алгоритм SART, основанный на реализации быстрого преобразования Хафа (БПХ).
Алгоритм решает задачу восстановления для случая зондирования монохроматическим параллельным излучением.
Требуемое доказательство асимптотической оценки сложности алгоритма представлено.
Численная реализация алгоритма проиллюстрирована результатами реконструкции модельного изображения размером 256 $\times$ 256 по полному набору 1021 углов БПХ и по разреженным синограммам, имеющим до 11 раз меньшее число проекционных углов.
\todo{доделать}

В \underline{\textbf{третьей главе}} предложена модель формирования измерений при наличии в объекте сильнопоглощающих включений.
Применяется метод квадратичного программирования для решения задачи восстановления в условиях предложенной модели.
Производятся исследования метода мягких ограничений на основе данных реальных экспериментальных измерений.

В \underline{\textbf{четвертой главе}} предложен алгебраический метод для восстановления томографических измерений при зондировании полихоматическим илучением.

В \underline{\textbf{заключении}} приведены основные результаты работы, которые заключаются в следующем:
%% Согласно ГОСТ Р 7.0.11-2011:
%% 5.3.3 В заключении диссертации излагают итоги выполненного исследования, рекомендации, перспективы дальнейшей разработки темы.
%% 9.2.3 В заключении автореферата диссертации излагают итоги данного исследования, рекомендации и перспективы дальнейшей разработки темы.
\begin{enumerate}
  \item Получено аналитическое выражение для операции транспонированного быстрого преобразования Хафа. Доказана теорема о симметричности четверти БПХ, отвечающей одной группе ориентаций лучей. Получен метод вычислительно эффективного вычисления транспонированного преобразования.
  \item Построен вычислительно эффективный алгебраический метод восстановления измерений компьютерной томографии на основе БПХ. Произведена адаптация метода для работы с данными реальных измерений. Изучены свойства пространства БПХ, влияние различной регуляризации на процесс восстановления.
  \item Построен метод восстановления рентгеновской томографии методом квадратичного программирования для объектов, содержащих сильнопоглощающие включения. На реальных экспериментальных измерениях проведено исследование метода мягких ограничений \todo{и сравнение с методом квадратичного программирования}
  \item Построен алгебраический алгоритм восстановления томографии в задаче зондирования полихроматическим излучением, основанный на вычиследнии взвешанных невязок по каждому из элементов, входящих в состав исследуемого образца. Предложены варианты регуляризации для востановления в полихроматической моде. \todo{описать, какой получен результат}
\end{enumerate}



\begin{comment}
%\newpage
При использовании пакета \verb!biblatex! список публикаций автора по теме
диссертации формируется в разделе <<\publications>>\ файла
\verb!../common/characteristic.tex!  при помощи команды \verb!\nocite! 
\end{comment}

\newpage
\ifdefmacro{\microtypesetup}{\microtypesetup{protrusion=false}}{} % не рекомендуется применять пакет микротипографики к автоматически генерируемому списку литературы
\ifnumequal{\value{bibliosel}}{0}{% Встроенная реализация с загрузкой файла через движок bibtex8
  \renewcommand{\bibname}{\large \authorbibtitle}
  \nocite{*}
  \insertbiblioauthor           % Подключаем Bib-базы
  %\insertbiblioother   % !!! bibtex не умеет работать с несколькими библиографиями !!!
}{% Реализация пакетом biblatex через движок biber
  \ifnumgreater{\value{usefootcite}}{0}{
%  \nocite{*} % Невидимая цитата всех работ, позволит вывести все работы автора
  \insertbiblioauthorcited      % Вывод процитированных в автореферате работ автора
  }{
  \insertbiblioauthor           % Вывод всех работ автора
%  \insertbiblioauthorgrouped    % Вывод всех работ автора, сгруппированных по источникам
%  \insertbiblioauthorimportant  % Вывод наиболее значимых работ автора (определяется в файле characteristic во второй section)
  \insertbiblioother            % Вывод списка литературы, на которую ссылались в тексте автореферата
  }
}
\ifdefmacro{\microtypesetup}{\microtypesetup{protrusion=true}}{}

