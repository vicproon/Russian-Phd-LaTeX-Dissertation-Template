\chapter{Задача реконструкции при зондировании полихроматическим излучением} \label{chapt3}
\section{Введение}

Цель данной главы - борьба с одним из артефактов восстановления, так называемым эффектом огрубления пучка или Beam Hardening. 
Причина возникночения таких артефактов - несоответствие излучения реального источника монохроматическому описанию, используемому в моделях для процедуры восстановления.
Другими словами, физическая модель формирования измерений, используемая при восстановлении, не соответствует действительности.
Во многих измерительных схемах, (альтернативно, нельзя поставить монохроматор) даже содержащих монохроматор []\todo{статья, подтверждающая наличие beam hardening даже при наличии монохроматора}, спектр излучения все равно полихроматический. 
При этом объект

Монохроматический коэффициент поглащения, доставляющий минимум целевой функции, будет содержать артефакты восстановления.
Они образуются из необходимости удовлетворить неправильным ограничениям, возникшим при упрощении физической модели.

Один из способов борьбы с данными артефактами - использование специальных []\todo{найти ссылку по поводу этого} регуляризаций или пост-обработки после восстановления.
Другой возможный способ - модифицировать механизм реконструкции таким образом, чтобы процедура находилась в соответствии с болле точной физиеской моделью.
В данном случае будет исследована возможность модификации метода восстановления, так, чтобы используемая физическая модель 
В работе восстановление производится в 2D-сечении и задача рассматривается в параллельном пучке.

\section{итерация алгебраического метода} \label{sect3_1}
\todo{текст ИТИС-2015}

%  В главе 2 будет представлена модель прямой проекции а так же алгебраический
%  метод для монохроматического источника излучения. Затем, в главе 3 будет
%  поставлена задача восстановления в монохроматическом пучке. Наконец, в гла-
%  ве 4 будут приведена соответствующая модификация алгебраического метода
%  для решения соответствующей задачи.

\begin{comment}
% Томография при Монохроматическом Излучении
% В текущей главе мы представляем алгебраический метод для случая монохро-
% матического излучения. Пусть f(x, y) – восстанавливаемое распределение ли-
% нейного коэффициента ослабления объекта. Пусть на объект падает параллель-
% ный пучок лучей интенсивностью 
%I 0 под углом φ. 
Тогда интенсивность излуче-
ния вдоль некоторого бесконечно тонкого луча со сдвигом s, %l(φ, s)
 после про-
хождения через объект будет
% I(l(φ, s)) = I 0 exp (− ∫ f(x, y)dl).
% (1)
Для краткости записи далее аргументы угла и сдвига, задающие прямую %l(φ, s)

будут опускаться. Данные значения подвергаются нормировке и логарифмиро-
ванию, в результате восстанавливается распределение линейного коэффициента
ослабления для энергии зондирующего излучения, путем решения системы
уравнений:
% ln (I 0 /I(l) ) = p(l) = ∫ f(x, y)dl
для всех углов измерения %φ
 и сдвигов s. Оператор интегрирования вдоль всех
возможных направлений называется преобразованием Радона. При восстановлении реальных измерений как входные данные% p(φ, s),
так и искомая характе-
ристика f(x, y) представляются дискретными изображениями размеров 
%N × M φ и N × N, 
соответственно, а преобразование Радона заменяется на преобра-
зование Хафа.
Для восстановления распределения линейного коэффициента ослабления ис-
пользуются в основном интегральные или алгебраические методы [4]. Особый
интерес представляют последние, так как их, как будет показано ниже, можно
использовать и для восстановления экспериментов, проводимых с использова-
нием полихроматического пучка. Пусть 
%f ∈ R N×N и p ∈ R N×M φ %
– соответ-
ственно входные и выходные данные, H – матрица линейного преобразования
Хафа размера 
%N 3 M φ ≈ N 4
 . Тогда томографическую проекцию можно записать
в виде СЛАУ
%p = Hf.
Решение этой СЛАУ методом наименьших квадратов в явном виде невозможно
ввиду огромного размера матрицы H, однако ее умножение на вектор f, как и
умножение транспонированной матрицы H T может быть посчитано быстро за
O(N 2 log N) используя методы обработки изображений [5]. Применяя метод
градиентного спуска для минимизации L2 нормы расхождения, получаем шаг
градиентного спуска
%f ξ = f ξ−1 + αH T (Hf ξ−1 − p), где
%α - параметр релаксации.
\end{comment}

\section{Модель томографии при немонохроматичесвом излучении}

В реальности спектр используемого для зондирования излучения не сингулярный, а коэффициент ослабления объекта меняется в зависимости от длины волны ($I_0 = I_0(\lambda), f(x, y) = f(x, y, \lambda)$).

Таким образом, при переходе к немонохроматическому случаю, уравнение \ref{eq:mono_fp} принимает следующий вид:

\begin{equation}
\label{eq:white_fp}
I(l) = \int_0^{+\infty}{\left\{
  I_0(\lambda) \exp{\left(-\int{f(x, y, \lambda) dl} \right) d\lambda} 
  \right\}}  
\end{equation}

Таким образом, оператор проецирования становится нелинейным.
При этом в случае, когда источник все же монохроматический, то есть $I_0(\lambda) = \delta(\lambda)$ легко видеть, что уравнение (\ref{eq:white_fp}) переходит обратно в (\ref{eq:mono_fp}).
Спектр используемого в экспериментах источника излучения можно измерить в лабораторных условиях и поэтому считается известным.
Переход к нелинейной задаче не был бы столь существенным, если бы при этом не терялась возможность восстановить зависимость подынтегральной функции от переменной интегрирования. 
Иными словами, без дополнительных знаний о структуре объекта, восстановить искомую функцию $f(x, y, \lambda)$ невозможно.

Предлагается использовать следующую модель формирования функции f %\cite{?}.
Будем считать, что исследуемый объект состоит из смеси K известных элементов, для которых известны их спектральные функции поглощения $f_k(\lambda)$.

Данные функции являются известными величинами, которые можно измерить в лабораторных условиях. В частности в данной работе были использованы значения, возвращаемые функциями библиотеки xraylib \cite{xraylib}.
При этом неизвестными будут пространственные распределения концентрации $c_k(x, y)$ каждого элемента.
\begin{equation}
 \notag
  f(x, y, \lambda) = \sum_{s=1}^K {c_s(x, y) \cdot f_s(\lambda)}
\end{equation}

Учитывая то, что интеграл внутри экспоненцирования – линейный оператор прямой проекции $H$, получим общее значение ослабления интенсивности входного излучения для смеси $c = (c_1, \dots, c_K)^T $ в пикселе измерения $j$ при линейном размере пикслея $\rho$:
\begin{equation}
  \label{eq:white_fp_final}
  I(c)_j = \int_0^{+\infty} {d\lambda \left\{
    I_0(\lambda) \exp{\left(
      -\sum_{k=1}^K {\rho f_k(\lambda) (H c_k)_j} 
      \right)}
  \right\}}
\end{equation}

Далее будет предложен итерационный процесс восстановления концентраций $с_k$, основанный на алгебраическом методе и минимизации $L_2$-нормы.

\section{Алгебраический Метод для Немонохроматического Случая}

В финале раздела будет выведена формула для расчета поправки на каждом шаге итерации метода восстановления концентраций $c_k$.
Пусть далее $i$ индексирует пиксели в пространстве исходных изображений-концентраций размера $N \times N$, $j$ индексирует пиксели в пространстве входных изображений-измерений размера $N \times M_\varphi$, а $k$, как и раньше, принимает значения $1 \dots K$ и индексирует различные элементы, составляющие исследуемый объект.
Пусть $h_{ij}$ – элементы матрицы прямой томографической проекции H.
Обозначим так же значения, измеренные детектором за $t$

Для того, чтобы выписать минимизационную процедуру необходимо составить функцию невязки.
По аналогии с выводом итерации алгебраического метода для монохроматичного случая, можно взять квадратичную невязку вида $(I(c) - t)^2$.
Однако эта величина размерная и зависит от суммарной мощности просвечивающего пучка.
Поэтому предлагается минимизировать невязку нормированных интенсивностей на величину $ S = \int_0^{+\infty}{I(\lambda)d\lambda}$.
Итак финальная функция для оптимизации имеет вид:
\begin{equation}
\label{eq:white_cost_function}
Q(c) = \left(\frac{I(c) - t}{S}\right)^2,
\end{equation} 
где за возведение в квадрат подразумевается обычная эвклидова норма разницы векторов размера $N \times M_\varphi$.

Для того, чтобы выстроить итерационный процесс вычисления концентраций, необходимо подсчитать градиент весовой функции по концентрациям каждого из элементов. 
Сделаем это покомпонентно:
\begin{equation}
  \notag
  \frac {\partial Q_j} {\partial c_{ki}} = 
  2 \frac {(I(c) - t)_j} {S} 
  \frac {\partial } {\partial c_{ki}}
  \left( \frac {I(c)_j} {S} \right).
\end{equation}

Рассчитаем возникшую частную производную:
\begin{equation}
  \notag
  \begin{split}
  \frac 1 S
  \frac {\partial I(c)_j} {\partial c_{ki}} &= 
  \frac 1 S
  \int_0^{+\infty} {d\lambda \left\{
    I_0(\lambda) 
    \frac \partial {\partial c_{ki}}
    \exp{\left(
      -\sum_{s=1}^K {\rho f_s(\lambda) (H c_s)_j} 
      \right)}
    \right\}} = \\
  &= 
  \int_0^{+\infty} {d\lambda \left\{
    -\rho f_k(\lambda) 
    \frac {I_0(\lambda)} {S}
    \exp{\left(
      -\sum_{s=1}^K {\rho f_s(\lambda) (H c_s)_j} 
         \right)}
    \frac {\partial (H c_k)_j} {\partial c_{ki}}
    \right\}} = \\
  &= 
  \frac {\mu_{kj}} {S} \frac {\partial (H c_k)_j} {\partial c_{ki}},
  \end{split}
\end{equation}

где введено обозначение $\mu_k$ для весовых коэффициентов, имеющих размерность синограммы и зависящих от элемента и прямой проекции:

\begin{equation}
  \label{eq:weights}
  \mu_{k} = \int_0^{+\infty} {d\lambda \left\{
    -\rho f_k(\lambda) 
    I_0(\lambda)
    \exp{\left(
      -\sum_{s=1}^K {\rho f_s(\lambda) (H c_s)} 
         \right)}
    \right\}}
\end{equation}

Эти коэффициенты должны учитывать спектральное взаимодействие различных элементов с источником, взвешивая невязку для каждого элемента, благодаря чему на каждой итерации вклад в разные концентрации будет разный.

Заметим, что производная $\frac {\partial (H c_k)_j} {\partial c_{ki}}$ соответствует обратной проекции в алгебраической процедуре восстановления. 
Таким образом, шаг итерации будет иметь вид
\begin{equation}
  \nabla_k \ Q = 2H^\intercal R_k \text{, где } R_{kj} = \frac {(I(c) - t)_j} {S} \mu_{kj}
\end{equation}

(6)
Где за R k обозначена взвешенная невязка для элемента k. Таким образом, мы
вычислили компоненты градиента по каждому из составляющих исходный объ-
ект элементу. Как видно из формулы (6), компоненты градиентов по различным
элементам могут вычисляться отдельно. Тем не менее, не стоит забывать, что
все элементы связаны через взвешенные невязки R k , в которых содержится
зависимость от всего вектора c. Итак, наконец, шаг итерации алгоритма выгля-
дит следующим образом:
\begin{equation}
  \label{white_iteration}
  c_k^\xi = c_k^{\xi - 1} - \gamma_k (\xi - 1) H^\intercal R_k(c_k^{\xi - 1}).
\end{equation}

% ξ
% ξ−1
% с k = c k
% 5
% − α k (ξ − 1)H T R k (c ξ−1 ) .
% (7)
Заключение
Была рассмотрена задача компьютерной томографии для восстановления изме-
рений, полученных при использовании немонохроматического источника излу-
чения. При этом привлекаются дополнительные знания относительно структуры
элементов, из которых состоит объект. Был явно выписан шаг итерации алгеб-
раического метода, восстанавливающего распределения концентраций состав-
ляющих объект элементов.
Для внесения ясности приведем псевдокод алгоритма восстановления:
Вход: %f k\delta(\lambda), I 0\delta(\lambda), t = I(φ, s)
Выход: %c k x, y)
Инициализация c
%Q <- (I(c) − t) 2
Пока Q > eps:
Вычисление весовых коэффициентов %μ jk по (5).
Вычисление взвешенных невязок R k по (6)
% ξ
Обновление c k по (7)
% Q <- (I(c) − t) 2
В дальнейшем мы планируем произвести оптимальную численную реализа-
цию алгоритма и проведение тестов как на модельных, так и на реальных дан-
ных, регистрируемых на лабораторном микротомографе, разработанном и
функционирующем в ИК РАН [6].

\section{численный эксперимент } \label{sect_3_2}
Одним из результатов данной работы является программная реализация и исследование работа алгоритмя восстановления, основанного на взешенных невязках.
