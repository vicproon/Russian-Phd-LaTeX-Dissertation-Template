\chapter{Восстановление в белом пучке} \label{chapt3}

\section{итерация алгебраического метода} \label{sect3_1}
\todo{текст ИТИС-2015}

ель данной работы – борьба с одним из таких артефактов (Beam Hardening
Artifact [3]), вызванным несоответствием излучения реального источника опи-
санию (монохроматическому), используемому в моделях для процедуры вос-
становления. В работе восстановление производится в 2D-сечении и задача рас-
сматривается в параллельном пучке.
В главе 2 будет представлена модель прямой проекции а так же алгебраический
метод для монохроматического источника излучения. Затем, в главе 3 будет
поставлена задача восстановления в монохроматическом пучке. Наконец, в гла-
ве 4 будут приведена соответствующая модификация алгебраического метода
для решения соответствующей задачи.
2
Томография при Монохроматическом Излучении
В текущей главе мы представляем алгебраический метод для случая монохро-
матического излучения. Пусть f(x, y) – восстанавливаемое распределение ли-
нейного коэффициента ослабления объекта. Пусть на объект падает параллель-
ный пучок лучей интенсивностью 
%I 0 под углом φ. 
Тогда интенсивность излуче-
ния вдоль некоторого бесконечно тонкого луча со сдвигом s, %l(φ, s)
 после про-
хождения через объект будет
% I(l(φ, s)) = I 0 exp (− ∫ f(x, y)dl).
% (1)
Для краткости записи далее аргументы угла и сдвига, задающие прямую %l(φ, s)

будут опускаться. Данные значения подвергаются нормировке и логарифмиро-
ванию, в результате восстанавливается распределение линейного коэффициента
ослабления для энергии зондирующего излучения, путем решения системы
уравнений:
% ln (I 0 /I(l) ) = p(l) = ∫ f(x, y)dl
для всех углов измерения %φ
 и сдвигов s. Оператор интегрирования вдоль всех
возможных направлений называется преобразованием Радона. При восстановлении реальных измерений как входные данные% p(φ, s),
так и искомая характе-
ристика f(x, y) представляются дискретными изображениями размеров 
%N × M φ и N × N, 
соответственно, а преобразование Радона заменяется на преобра-
зование Хафа.
Для восстановления распределения линейного коэффициента ослабления ис-
пользуются в основном интегральные или алгебраические методы [4]. Особый
интерес представляют последние, так как их, как будет показано ниже, можно
использовать и для восстановления экспериментов, проводимых с использова-
нием немонохроматического пучка. Пусть 
%f ∈ R N×N и p ∈ R N×M φ %
– соответ-
ственно входные и выходные данные, H – матрица линейного преобразования
Хафа размера 
%N 3 M φ ≈ N 4
 . Тогда томографическую проекцию можно записать
в виде СЛАУ
%p = Hf.
Решение этой СЛАУ методом наименьших квадратов в явном виде невозможно
ввиду огромного размера матрицы H, однако ее умножение на вектор f, как и
умножение транспонированной матрицы H T может быть посчитано быстро за
O(N 2 log N) используя методы обработки изображений [5]. Применяя метод
градиентного спуска для минимизации L2 нормы расхождения, получаем шаг
градиентного спуска
%f ξ = f ξ−1 + αH T (Hf ξ−1 − p), где
%α - параметр релаксации.
3
Прямая Проекция при Немонохроматическом Источнике
В реальности спектр используемого для зондирования излучения не сингуляр-
ный, а коэффициент ослабления объекта меняется в зависимости от длины вол-
ны 
%(I 0 = I 0 (λ), f(x, y) = f(x, y, λ)).%
 Таким образом, при переходе в немоно-
хроматический случай, уравнение (1) принимает следующий вид:
% +∞
% I(l) = ∫ 0
% I 0 (λ) exp(− ∫ f(x, y, λ)dl) dλ .
% (2)
Таким образом, оператор проецирования становится нелинейным. При этом, в
случае когда источник все же монохроматический, то есть
 %I 0 (λ) = δ(λ), %
 легко
видеть, что уравнение (2) переходит обратно в (1). Спектр используемого в экс-
периментах источника излучения мы считаем известным.
Переход к нелинейной задаче не был бы столь существенным, если бы при этом
не терялась возможность восстановить зависимость подынтегральной функции
от переменной интегрирования. Иными словами, без дополнительных знаний о
структуре объекта, восстановить искомую функцию
% f(x, y, λ) %
невозможно.
538Предлагается использовать следующую модель формирования функции f [7].
Будем считать, что исследуемый объект состоит из смеси K известных элемен-
тов, для которых известны их спектральные функции поглощения 
%f k (λ). 
Дан-
ные функции являются известными величинами, которые можно измерить в
лабораторных условиях. При этом неизвестными будут пространственные рас-
пределения концентрации c k (x, y) каждого элемента.
%f(x, y, λ) = ∑ K
%k=1 c k (x, y) ∗ f k (λ).
%(3)
Учитывая то, что интеграл внутри экспоненцирования – линейный оператор H,
получим общее значение ослабления интенсивности входного излучения для
смеси c = (c 1 , ... , c K ) T :
% +∞
% I(c) = ∫ 0
% I 0 (λ)exp(− ∑ K
% k=0 f k (λ)Hc k )dλ .
Далее будет предложен итерационный процесс восстановления концентраций
с k , основанный на алгебраическом методе и минимизации L2-нормы.
4
Алгебраический Метод для Немонохроматического
Случая
В финале раздела будет выведена формула для расчета поправки на каждом
шаге итерации метода восстановления концентраций c k .
Пусть далее i индексирует пиксели в пространстве исходных изображений-
концентраций размера N x N, j индексирует пиксели в пространстве входных
изображений-измерений размера % N × M φ 
, а k, как и раньше, принимает значе-
ния 1. . K и индексирует различные элементы, составляющие исследуемый объ-
ект. Пусть h\_ij – элементы матрицы H. Введем невязку -
%Q(c) = (I(c) − t) 2 ,
%N×M
%φ
%где t ∈ R
– значения, зарегистрированные детектором. Для того, чтобы
выстроить итерационный процесс вычисления концентраций, необходимо под-
считать градиент невязки. Сделаем это покомпонентно:
% ∂Q j
% ∂c ki
% = 2(I(c) − t) j
% ∂I(c) j
% ∂c ki
% .
Рассчитаем возникшую частную производную:
% ∂I(c) j
% ∂c ki
% +∞
% +∞
% = ∫ 0 dλ {
% ∂
% ∂c ki
% exp(− ∑ s f s (λ)[Hc s ] j )} =
% ∂
% ∫ 0 dλ {exp(− ∑ s f s (λ) [Hc s ] j ) ∂c i (−f k (λ)[Hc k ] j )} =
% k
% ∂[Hc k ] j
% +∞
% ∫ 0 dλ {−f k (λ) exp(− ∑ s f s (λ) [Hc s ] j ) ∂c i },
% k
% 539
% (4)То есть имеем
% ∂I(c) j
% ∂c ki
% +∞
% = μ jk h ji , где μ jk = ∫ 0 {−f k (λ) exp(− ∑ s f s (λ) [Hc s ] j )}dλ .
(5)
В уравнении (5) введено новое обозначение, некоторый весовой коэффициент,
одинаковый для всех пикселей входных данных. Подставляя (5) в (4) получим
выражение для градиента общей невязки решаемой минимизационной задачи:
% ∇ k Q = 2H T R k , где R jk = (I(c) − t) j μ jk ,
(6)
Где за R k обозначена взвешенная невязка для элемента k. Таким образом, мы
вычислили компоненты градиента по каждому из составляющих исходный объ-
ект элементу. Как видно из формулы (6), компоненты градиентов по различным
элементам могут вычисляться отдельно. Тем не менее, не стоит забывать, что
все элементы связаны через взвешенные невязки R k , в которых содержится
зависимость от всего вектора c. Итак, наконец, шаг итерации алгоритма выгля-
дит следующим образом:
% ξ
% ξ−1
% с k = c k
% 5
% − α k (ξ − 1)H T R k (c ξ−1 ) .
% (7)
Заключение
Была рассмотрена задача компьютерной томографии для восстановления изме-
рений, полученных при использовании немонохроматического источника излу-
чения. При этом привлекаются дополнительные знания относительно структуры
элементов, из которых состоит объект. Был явно выписан шаг итерации алгеб-
раического метода, восстанавливающего распределения концентраций состав-
ляющих объект элементов.
Для внесения ясности приведем псевдокод алгоритма восстановления:
Вход: %f k (λ), I 0 (λ), t = I(φ, s)
Выход: %c k (x, y)
Инициализация c
%Q <- (I(c) − t) 2
Пока Q > eps:
Вычисление весовых коэффициентов %μ jk по (5).
Вычисление взвешенных невязок R k по (6)
% ξ
Обновление c k по (7)
% Q <- (I(c) − t) 2
В дальнейшем мы планируем произвести оптимальную численную реализа-
цию алгоритма и проведение тестов как на модельных, так и на реальных дан-
ных, регистрируемых на лабораторном микротомографе, разработанном и
функционирующем в ИК РАН [6].

\section{численный эксперимент } \label{sect_3_2}
