\chapter*{Введение}							% Заголовок
\addcontentsline{toc}{chapter}{Введение}	% Добавляем его в оглавление

\newcommand{\actuality}{}
\newcommand{\progress}{}
\newcommand{\aim}{{\textbf\aimTXT}}
\newcommand{\tasks}{\textbf{\tasksTXT}}
\newcommand{\novelty}{\textbf{\noveltyTXT}}
\newcommand{\influence}{\textbf{\influenceTXT}}
\newcommand{\methods}{\textbf{\methodsTXT}}
\newcommand{\defpositions}{\textbf{\defpositionsTXT}}
\newcommand{\reliability}{\textbf{\reliabilityTXT}}
\newcommand{\probation}{\textbf{\probationTXT}}
\newcommand{\contribution}{\textbf{\contributionTXT}}
\newcommand{\publications}{\textbf{\publicationsTXT}}

{\actualityandprogress}. Методы восстановления измерений в задаче компьютерной томографии можно разделить на интегральные и алгебраические. К интегральным относится метод светки и обратной проекции....

{\aim} ~данной работы являются разработка метода реконструкции, позволяющего учесть присутствие в объекте сильнопоглощающих включений, а так же метода численной интерпретации результатов измерений многокомпонентных объектов.

Для достижения поставленной цели были решены следующие {\tasks}:
\begin{enumerate}
  \item построен асимптотически быстрый алгебраический метод реконструкции, основанный на применении быстрого преобразования Хафа.
  \item доказана сходимость построеного алгебраического метода реконструкции, за счет полученного математического выражения градиента быстрого преобразования Хафа.
  \item построен алгоритм реконструкции для объектов, содержащих сильнопоглощающие включения.
  \item построен алгоритм реконструкции, учитывающий покомпонентное ослабление полихроматического спектра.
\end{enumerate}

{\novelty}
\begin{enumerate}
  \item Впервые для реконструкции томографических измерений было применено быстрое преобразование Хафа.
  \item Впервые получено выражение для производной быстрого преобразования Хафа, а так же алгоритм его эффективного вычисления.
  \item Построен алгоритм реконструкции, учитывающий вклад сильно поглощающих включений с помощью оригинальной модели ограничений-неравенств.
  \item Предложена схема обработки данных полихроматического зондирования, при которой восстанавливаются реальные физические концентрации элементов.
\end{enumerate}

{\influence} ~Результаты, полученные в диссертационной работе, используются для обработки данных лабораторных исследований. Построенные алгоритмы лягут в основу программного обеспечения новых моделей промышленных томографов.

Полученное в работе выражение для градиента быстрого преобразования Хафа имеет общетеоретическое значение и уже применяется в области машинного обучения для обратного распространения ошибки в нейронных сетях глубокого обучения через слой БПХ.

{\methods}
Для решения задач реконструкции томографических измерений используются методы теории условной и безусловной оптимизации: градиентные методы оптимизации, квадратичное программирование, регуляризация.
Для ускорения итерации алгебраического метода используются алгоритмы обработки изображений в виде быстрого преобразования Хафа.


{\defpositions}
\begin{enumerate}
  \item Предложен эффективный вычислительный метод решения задачи томографической реконструкции FHT-SIRT, основанный на алгебраическом подходе, который позволяет снизить асимптотическую оценку сложности вычисления итерации с $O(n^3)$ до $O(n^2~\log n)$, что подтвержается численными экспериментами и замерами времени работы программной реализации алгоритма.
  \item Проведено математическое обоснование сходимости предложенного метода.
  \item Предложен метод реконструкции на основе квадратичного программирования, который позволяет уменьшить артефакты на восстановленном изображении, вызванные наличием сильно поглощающих областей в зондируемом объекте.
  \item Предложен алгебраический метод реконструкции для случая полихроматического зондирования, который решает оптимизационную задачу реконструкции относительно линейной комбинации концентраций с ограничениями-неравенствами на их область значений.
\end{enumerate}


{\reliability} полученных результатов обеспечивается модельными экспериментами и численными симуляциями, а так же экспериментами с восстановлением реально измеренных в лабораторных условиях образцов.\ Результаты находятся в соответствии с результатами, полученными другими авторами.


{\probation}
Основные результаты работы докладывались~на: конферециях 
35-я конференция молодых ученых и специалистов «Информационные технологии и системы» (2012, 19 - 25 августа, Петрозаводск, Россия),
11th Biennal Conference on High Resolution X-Ray Diffraction and Imaging (XTOP 2012, St. Petersburg, Russia), 
29th European Conference on Modelling and Simulation (ECMS 2015, Albena, Bulgaria),
Eighth International Conference on Machine Vision (ICMV 2015, Barcelona, Spain),
на общефизическом семинаре ИПТМ РАН (октябрь 2016).

{\contribution} Все результаты диссертации, вынесенные на защиту, получены автором самостоятельно.
Автором самостоятельно реализованы методы восстановления FHT-SIRT из первой главы, барьерных функций из второй, метод взвешанных невязок из третьей, проведены численные экспериметны по обработке реальных и модельных данных.
Постановка задач и обсуждение результатов проводились совместно с научным руководителем.
Генерация модельных данных для экспериментов в полихроматике проводилась аспирантом факультета КН НИУ ВШЭ Ингачевой А.~С. 
Программная имплементация метода мягких ограничений, использованная для сравнения с методом барьерных функций во второй главе, принадлежит Соколову В.~В.
Измерения для экспериментов по восстановлению зуба со свинцовым включением производились на лабоработном источнике ИК РАН в лаборатории рефлектометрии и малоуглового рассеяния.
Многие аспекты исследований в разное время обсуждались с Чукалиной М.~В., Николаевым Д.~П., Бузмаковым А.~В., Ингачевой А.~С., Соколовым В.~В.


\ifthenelse{\equal{\thebibliosel}{0}}{% Встроенная реализация с загрузкой файла через движок bibtex8
    \publications\ Основные результаты по теме диссертации изложены в 11 печатных изданиях, 
    3 из которых изданы в журналах, рекомендованных ВАК, 
    6 "--- в тезисах докладов.%
}{% Реализация пакетом biblatex через движок biber
%Сделана отдельная секция, чтобы не отображались в списке цитированных материалов
  \begin{refsection}%
    \printbibliography[heading=countauthorvak, env=countauthorvak, keyword=biblioauthorvak, section=1]%
    \printbibliography[heading=countauthornotvak, env=countauthornotvak, keyword=biblioauthornotvak, section=1]%
    \printbibliography[heading=countauthorconf, env=countauthorconf, keyword=biblioauthorconf, section=1]%
    \printbibliography[heading=countauthor, env=countauthor, keyword=biblioauthor, section=1]%

    \publications\ Основные результаты по теме диссертации изложены в \arabic{citeauthor} печатных изданиях \nocite{PruBuzNik13, Prun2013Crys, Vestnik2016, Prun2013AutomAndRemCont, buz_jac_2015, chukalina2014xray}, 
    \arabic{citeauthorvak} из которых изданы в журналах, рекомендованных ВАК %\nocite{PruBuzNik13, Prun2013Crys, Vestnik2016}, 
    \arabic{citeauthorconf} "--- в тезисах докладов \nocite{itas2015Prun,itas2015Ingacheva,ecms2015Chukalina, icmv2015Chukalina, embc2013Buzmakov, nikolaevfast}.
  \end{refsection}
} % Характеристика работы по структуре во введении и в автореферате не отличается (ГОСТ Р 7.0.11, пункты 5.3.1 и 9.2.1), потому её загружаем из одного и того же внешнего файла, предварительно задав форму выделения некоторым параметрам

\textbf{Объем и структура работы.} Диссертация состоит из~введения, четырёх глав, заключения и~двух приложений.
%% на случай ошибок оставляю исходный кусок на месте, закомментированным
%Полный объём диссертации составляет  \ref*{TotPages}~страницу с~\totalfigures{}~рисунками и~\totaltables{}~таблицами. Список литературы содержит \total{citenum}~наименований.
%
Полный объём диссертации составляет
\formbytotal{TotPages}{страниц}{у}{ы}{}, включая
\formbytotal{totalcount@figure}{рисун}{ок}{ка}{ков} и
\formbytotal{totalcount@table}{таблиц}{у}{ы}{}.   Список литературы содержит  
\formbytotal{citenum}{наименован}{ие}{ия}{ий}.

\actuality



Для успешного решения многих прикладных задач необходимой является возможность подробного исследования внутренней структуры образцов без их физического изменения или разрушения.
Речь идет о таких областях как медицина, промышленность, биология, геология. 
Методы, позволяющие осуществлять такие исследования, назыавются томографией.

Современное развитие вычислительной техники, повышение ее доступности, появление новых инструментариев обработки информации, развитие алгоритмов и компьютерных наук делают возможными применение в анализе данных исследований алгоритмов обработки изображений.
В частности, в работе рассматриваются методы обработки данных, возникающих после экспериментов и измерений рентгеновской томографии.
Эти данные позволяют восстановить внутренние характеристики объекта, а именно пространственное распределение коэффициента ослабления рентгеновского излучения, после измерения ослабления излучения, измеренного под разными углами.

Центральную роль в подобных измерениях играет томограф - программно-аппаратный комплекс, в котором совмещены как экспериментальная часть измерения, так и математическая обработка полученных данных.
Аппаратная часть состоит в определении поглащения объектом испускаемого рентгеновского излучения под разными углами проекции.
При этом возможно много различных конфигураций подобного эксперимента: форма исходящего пучка, спектр излучения, спектр чувствительности и форма детектора, взаимное расположение детектора, источника и оси вращения объекта.
При этом в приложениях медицины критическую роль играет доза облучения, которой подвергается объект.

Сырые данные, полученные непосредственно после измерений не пригодны для их анализа экспертом.
Для того чтобы получить из данных эксперимента, так же называемых синограммой, искомые пространственные распределения характеристик объектов, необходимо произвести дополнительную обработку, называемую методами восстановления компьютерной томографии.
Алгоритмы восстановления и составляют программную часть томографа.



1. что такое томография - позволяет без разрушения восстановить внутреннюю структуру в ppt посмотри о видах томографии	

2. что такое томограф - это аппаратно-программный комплекс, ты будешь в диссере рассматривать проблемы, возникающие в программной части

3. из Лешиных тезисов - разные объекты требуют разного при реконструкции - отсюда разное качество реконструкции при использовании разных методов реконструкции

4.но не только методы формируют ошибки результатов восстановления - анализ других источников

5. описываем самое простое измерение - параллельная схема, монохроматический пучок - здесь первое короткое описание модели формирования сигнала в измерительной схеме

5. приходим к выводу, что алгебраические имеют преимущества над интегральными, поскольку можно учесть и влияние аппаратуры и работать при наличии большого шума
но алгебраические медленные - поэтому надо развивать быстрые реализации, хаф

6. усложним задачу. есть сильнопоглощающие включения - часть ко введению из ecms 2015, icmv 2015 подвести к выводу, что надо создавать и в этом простом для измерения случае (параллельный пучок и монохроматика) принципиально новые методы

7. еще усложним задачу - монохроматику заменим полихроматикой и обретем радости с beam-hardening 2 пути решать - честно или постобработкой измеренных проекций или самой восстановленной картинки.

все это дело объединяем еще раз коротко и выводим актуальность работы.

\section{Обзор литературы}

\todo{введение - обзор из статьи аит2013 и бакалаврского диплома}

Компьютерная томография --- это инструмент, широко применяемый сегодня в различных областях жизни, таких как неразрушающий контроль в промышленности, диагностика и лечение  в медицине и др. Целью применения метода компьютерной томографии является реконструкция пространственного распределения характеристик объекта без его физического разрушения. Однако сырые данные, регистрируемые в ходе измерения высокотехнологичными томографическими системами, сами по себе не являются достаточными для описания объекта исследования, а требуют привлечения дополнительного математического аппарата (методов реконструкции изображений, методов визуализации восстановленных изображений и др.). 

Задача восстановления изображения объекта по набору зарегистрированных томографических проекций известна как задача обращения преобразования Радона при условии конечного числа направлений. Методы ее решения, интегральные \cite{herman2013mathematical} и алгебраические \cite{algebraic_methods}, постоянно совершенствуются. Предлагаются новые версии алгоритмов, основанных на алгебраическом подходе, способных работать с сильно зашумлёнными проекциями. Такое условие сформировано необходимостью сокращать время регистрации проекций. Для некоторых применений уменьшение времени регистрации связано с требованием сокращения дозы облучения, для других --- обусловлено высокой динамикой поведения исследуемого объекта. Также следует отметить, что алгебраические методы реконструкции незаменимы, когда речь идет об экспериментах с малым числом проекционных углов и измерениях в ограниченном телесном угле. Только алгебраические методы применимы для решения задач трансмиссионно-эмиссионной томографии, если ослаблением зондирующего и вторичного излучений пренебречь нельзя.

Хотя концептуально алгебраические методы проще интегральных и лучше справляются с наличием высокого уровня шума в проекциях, они проигрывают последним по времени реконструкции. Численные реализации новых быстрых алгоритмов и применение распараллеливания на базе графических процессоров делает алгебраический подход конкурентноспособным с сохранением всех преимуществ его поведения в среде с высоким уровнем шума.

\todo{тут идет копипаст из статьи Соколова. надо перевести и переформулировать.}

Appearance of metal streak artifacts in reconstructed CT images is a known problem \cite{barrett2004artifacts, boas2012ct, nasirudin2015reduction,
park2015computed}. Such artifacts are caused by multiple reasons, including beam hardening, scatter, Poisson noise, motion, and edge effects \cite{boas2012ct}. A variety of tricks are used to avoid these artifacts: hardware tricks including automatic control of the X-ray tube voltage and current modulations, software preprocessing of the projection data before reconstruction - sinograms are filtered using adaptive methods of filtering. For example an adaptive expansion of the detector element size in regions of photon starvation are used \cite{boas2012ct}. One more group of methods carries out measurements using a multi-energy scan \cite{bamberg2011metal}. Also there are methods in which
the core of the reconstruction method takes into account the
mathematical model of a sinogram formation. The statistical reconstruction techniques are used to deal with the metal artifact reduction problem \cite{jmuller2006, buzug2008computed}. Contrary to filtered back-projection group of methods, in those techniques, the influence of each single beam on the image reconstruction can be weighted separately. The maximum likelihood
(MLEM) algorithm \cite{buzug2008computed} and modified MLEM algorithm called $\lambda$-MLEM \cite{oehler2007statistical} improve image quality comparing to pure interpolation or missing data concept \cite{amirkhanov2012evaluation}.