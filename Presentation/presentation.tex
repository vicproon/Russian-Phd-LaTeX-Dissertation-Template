\documentclass[12pt]{beamer}
\usepackage[T2A]{fontenc}
\usepackage[utf8]{inputenc}
\usepackage[english,russian]{babel}
\usepackage{amssymb,amsfonts,amsmath,mathtext}
\usepackage{cite,enumerate,float,indentfirst}
\usepackage{comment}

\graphicspath{{../images/}{images/}{../Dissertation/images}} 



% \usetheme[secheader]{Boadilla}
% \usecolortheme{seahorse}

\usetheme{Pittsburgh}
\usecolortheme{whale}

\beamertemplatenavigationsymbolsempty

\newcommand{\todo}{\alert}
\input{../common/data.tex}      % Основные сведения
\input{../common/newnames.tex} 

\setbeamercolor{footline}{fg=blue}
\setbeamertemplate{footline}{
  \leavevmode%
  \hbox{%
  \begin{beamercolorbox}[wd=.333333\paperwidth,ht=2.25ex,dp=1ex,center]{}%
    % И. О. Фамилия, Организация кратко
    \thesisAuthorShort, \thesisOrganizationShort
  \end{beamercolorbox}%
  \begin{beamercolorbox}[wd=.333333\paperwidth,ht=2.25ex,dp=1ex,center]{}%
    % Город, 20XX
    \thesisCity, \thesisYear
  \end{beamercolorbox}%
  \begin{beamercolorbox}[wd=.333333\paperwidth,ht=2.25ex,dp=1ex,right]{}%
  Стр. \insertframenumber{} из \inserttotalframenumber \hspace*{2ex}
  \end{beamercolorbox}}%
  \vskip0pt%
}

\newcommand{\itemi}{\item[\checkmark]}

\newcommand{\rom}[1]{%
  \textup{\uppercase\expandafter{\romannumeral#1}}%
}

\AtBeginSection[]{
  \begin{frame}
  \vfill
  \centering
  \begin{beamercolorbox}[sep=8pt,center,shadow=true,rounded=true]{title}
    \usebeamerfont{title}\insertsectionhead\par%
  \end{beamercolorbox}
  \vfill
  \end{frame}
}

%\title{\small{Название презентации}}
\title{\small{\thesisTitle}}
\author{\small{%
\emph{Выступающий:}~\thesisAuthorShort\\%
\emph{Руководитель:}~\supervisorRegaliaShort~\supervisorFioShort}\\%
\vspace{30pt}%
\thesisOrganization%
\vspace{20pt}%
}
\date{\small{\thesisCity, \thesisYear}}

\begin{document}

\maketitle

\begin{frame}
\frametitle{Объект исследования}
\textbf{Компьютерная томография} --- 

неинвазивный неразрушающий метод исследования внутренней структуры образцов.
\\ 
Применятеся для:
\begin{itemize}
  \item Контроль качества изготавливаемых изделий
  \item Медицинская диагностика
  \item Численное моделирование на результатах КТ
  \item Биологические и геологические исследования
  \item Изготовление прототипов (3д-печать) по результатам КТ (?)
\end{itemize}

\end{frame}

\begin{frame}
\frametitle{Объект исследования}
\centering
Томографическое измерение:
\begin{tabular}{p{0.45\textwidth} p{0.55\textwidth}}
  \begin{figure}[H]
    \includegraphics[width=0.45\textwidth]{../Dissertation/images/part1_img/experiment}
  \end{figure}
  &
  \begin{itemize}
  \item $N$ ячеек детектора
  \item $N_\varphi$ углов сканирования
  \item Для каждого угла $\varphi$ и каждой ячейки $s$ измеряется интенсивонсть прошедшего рентгеновского излучения \\
    $\mathrm I \left( \varphi, s \right) = \mathbb P (f(x, y))$
  \end{itemize}
\end{tabular}
\end{frame}

\begingroup
% \small

\begin{frame}
\frametitle{Предмет исследования}
Компьютерная томография --- программно-аппаратный комплекс
% \pause
\setlength{\leftmargini}{0em}
\begin{columns}[T,onlytextwidth]
  \begin{column}{.4\textwidth}
  \begin{itemize}
    \item экспериментальная схема
    \item калибровка 
    \item \textbf{программная обработка $\rightarrow$}
  \end{itemize}
  \end{column}
  \pause
  \begin{column}[t]{0.4\linewidth}
  \begin{itemize}
    \item предобработка
    \item \textbf{алгоритмы восстановления $\rightarrow$}
    \item постобработка
  \end{itemize}
  \end{column}
  \pause
  \begin{column}[t]{0.35\linewidth}
  \begin{itemize}
    \item \textbf{метод оптимизации}
    \item \textbf{модель регуляризации}
    \item \textbf{физическая модель}
    \item реализация математических примитивов
  \end{itemize}
  \end{column}
\end{columns}
\end{frame}
\endgroup

\begin{frame}
\frametitle{Предмет исследования}
\framesubtitle{Предположения}
\begin{itemize}
  \item Рассматривается плоское сечение
  \item Параллельная схема проекции
  \item Ослабление интенсивности излучения подчиняется закону Бугера-Ламберта-Бера:
\end{itemize}
  $$
  \mathrm I \left( \varphi, \xi , \lambda \right) = \mathrm I_0(\lambda) \exp\left( {-\int_{l(\varphi, \xi)} \! f(l, \lambda) \mathrm d l }\right),
  $$
  \small
  $f(x, y)$ ---  описывает распределение линейного коэффициента ослабления рентгеновского излучения \\
  $\mathrm I(\varphi, \xi, \lambda)$ --- зарегистрированная детектором интенсивность излучения длины волны $\lambda$ \\
  $\mathrm I_0(\lambda)$ --- интенсивность зондирующего излучения \\
  $l(\varphi, \xi)$ --- параметризация прямой под углом $\varphi$, и сдвигом $\xi$ \\
\end{frame}

\begin{frame}
\frametitle{Задача восстановления}
  Связь линейного коэффициента ослабления и интенсивности описывается преобразованием Радона $R[f(x,y)](\varphi, \xi)$:

  $$
  R[f](\varphi, \xi) = 
 \iint \! \mathrm d x \mathrm d y f(x,y)\delta(x\cos\varphi + y\sin\varphi - \xi)
  $$


  После логарифмирования закона ослабления:
  $$
  \ln \left (\frac{\mathrm I_0(\lambda)}{\mathrm I(\varphi, \xi, \lambda)} \right) = p(\varphi, \xi) = R[f](\varphi, \xi)
  $$

\end{frame}

\begin{frame}
\frametitle{Задача восстановления}
\begin{figure}
\centering
    прямая задача, модель измерения\\
    $\rightarrow$
    \\
    \includegraphics[width=0.8\textwidth]{sl_sinogram}
    \\
    $\leftarrow$ \\
    обратная задача, процедура восстановления
\end{figure}
\end{frame}

% \begin{frame}
% \frametitle{Задача восстановления}
% 
% Задача восстановления функции $f(x,y)$- задача обращения преобразования Радона полученных измерений $p$:
% $$
% f(x,y) = R^{-1}(p(\varphi, \xi))
% $$
% \end{frame}

\begingroup
\small
\begin{frame}
\frametitle{Методы восстановления}
\begin{tabular}{p{0.15\textwidth} | p{0.4\textwidth} | p{0.4\textwidth}}
\hspace{-1cm} семейство & Интегральные & Алгебраические \\ \hline
\hspace{-1cm} подход & Аналитическая формула для обращения $R^{-1}[p(\varphi, \xi)](x,y)$ & Оптимизационная задача $\min_{f} \Norm{R[f] - p}$\\ \hline
\hspace{-1cm} представители & FBP & ART, SART, SIRT \\ \hline
\hspace{-1cm} сложность & $O(N^2 \log N)$ & $O(N^3)$ \\ \hline
\hspace{-1cm} особенности & универсальность & возможность учета \hspace{1cm} модели объекта или измерительной схемы \\
                          & требует полного набора проекционных углов, & \\ 
                          & их равномерного распределения & \\
                          & чувствительны к шумам & \\
\end{tabular}

$N$ --- число ячеек детектора
\end{frame}
\endgroup


\begin{frame}
\frametitle{Цели работы}

разработка, теоретическое обоснование и численное исследование новых алгоритмов компьютерной томографии для восстановления измерений при зондировании рентгеновским моно- и полихроматическим излучением.
\end{frame}

\begin{frame}
\frametitle{Цели работы}
\begin{itemize}
\item разработка и реализация асимптотически быстрых алгебраических методов реконструкции в задаче компьютерной томографии
\item разработка новых математических моделей для решения задач восстановления при наличии неисключаемых причин ошибок
\item разработка и численное тестирование алгоритмов восстановления в контексте построенных математических моделей
\end{itemize}
\end{frame}

\begin{frame}
\frametitle{Задачи}

\begin{itemize}
  \item построение асимптотически быстрого алгебраического метода реконструкции, основанного на применении быстрого преобразования Хафа
  \item доказательство сходимости построенного алгебраического метода реконструкции
  \item построение алгоритма реконструкции для объектов, содержащих сильнопоглощающие включения
  \item построение алгоритма реконструкции, учитывающего покомпонентное ослабление полихроматического спектра.
\end{itemize}

\end{frame}

\begingroup
\small
\begin{frame}
\frametitle{Положения, выносимые на защиту}
\begin{enumerate}
  \item Предложен алгебраический метод восстановления FHT-SIRT, который позволяет снизить асимптотическую оценку сложности вычисления итерации с $O(n^3)$ до $O(n^2~\log n)$, что подтвержается численными экспериментами и замерами времени работы программной реализации алгоритма.
  \item Доказана сходимость построенного алгебраического метода реконструкции. 
  \item Предложен метод реконструкции на основе квадратичного программирования, который позволяет исправить артефакты восстановления, вызванные наличием в исследуемых объектах сильнопоглощающих включений.
  \item Построен метод восстановления при зондировании полихроматическим излучением, позволяющий оценить распределение концентраций разных элементов в объекте.
\end{enumerate}
\end{frame}
\endgroup

\section{Вычислительно эффективный алгебраический метод восстановления FHT-SIRT}

\begin{frame}
\frametitle{Алгебраический метод}
\framesubtitle{Основные положения}
\begin{itemize}
  \item непрерывные функции $\rightarrow$ дискретные изображения:

    {
    \centering
    $f(x,y) farrow f_i,\ p(\varphi, s) \rightarrow p_j$
    \par
    }
  \vspace{0.5cm}
  \item преобразование Радона $\rightarrow$ преобразование Хафа:
  
    {
    \centering
    $R[f](\varphi, s) \rightarrow (\mathrm W f)_j$ 
    \par
    }
  \vspace{0.5cm}

    $\mathrm W$ - матрица проекции, указывает вклад пикселя $i$ в лучевую сумму вдоль луча $j$.
    Разреженная матрица размера $N_\varphi * N^3$, в которой только порядка $O(N_\varphi * N)$ ненулевых элементов
    \vspace{0.5cm}
  \item решение разреженной СЛАУ большой размерности итерационным методом

    {
    \centering
    $p = \mathrm W f$
    \par
    }

\end{itemize}
\end{frame}

\begin{frame}
\frametitle{Алгебраический метод}
\framesubtitle{Решение СЛАУ}
\centering
$\Norm{p - \mathrm W f} \rightarrow \min\limits_f$

Оптимизационная задача решается итерационным методом, шаг итерации имеет вид
\vspace{0.5cm}

\begingroup
\footnotesize

\hspace*{-0.5cm}
\begin{tabular}{c|c|c}
ART & SART & SIRT \\ \hline
для каждого луча & для каждого угла & для всех лучей\\
$j = 1 \dots N * N_\varphi$ & $\varphi_k$ & \\
$\hat{f} = f - \gamma \mathrm W^{\mathrm T}_j(\mathrm W f - p)$ &
$\hat{f} = f - \gamma \mathrm {W^{\varphi_k}}^{\mathrm T}(\mathrm W f - p)$ &
$\hat{f} = f - \gamma \mathrm W^{\mathrm T}(\mathrm W f - p)$ \\
\end{tabular}

\vspace{0.5cm}
\raggedright
\endgroup

$\mathrm W_j$ --- столбец матрицы для луча $j$,\\
$\mathrm W^\varphi$ ---  матрица проекции на угол $\varphi$, $\mathrm W = \sum_\varphi {\mathrm W^\varphi}$


\end{frame}

\begin{frame}
\frametitle{Быстрое преобразование Хафа}
\begin{columns}[T,onlytextwidth]
  \hspace*{-0.5cm}
  \begin{column}{0.65\textwidth}
  Приближенный способ вычисления сумм интенсивностей изображения вдоль всевозможных приямых
  \begin{figure}
    \includegraphics[width=1\textwidth]{fht}
  \end{figure}
  \end{column}
  \begin{column}{0.45\textwidth}
  \begin{itemize}
    \item диадические паттерны суммирования
    \item рекурсивная процедура построения
    \item для больших размеров изображения хорошо приближает прямые (отклонение не превышает  $\frac 1 6 \log N$ ) %\cite{ershov2015dyadic})
    \item асимптотическая сложность $N^2 \log N$
  \end{itemize}
  \end{column}
\end{columns}

\end{frame}

\begin{frame}
\frametitle{Быстрое преобразование Хафа}
\frametitle{Процедура формирования диадических паттернов}
  \begin{figure}
  \centering
    \includegraphics[width=1\textwidth]{../Dissertation/images/part1_img/hough_proc}
  \end{figure}
\end{frame}

\begin{frame}
\frametitle{FHT-SIRT}
\framesubtitle{прямая проекция}
\begin{columns}[T,onlytextwidth]
  \hspace*{-0.5cm}
  \begin{column}{0.65\textwidth}
  \begin{figure}
    \includegraphics[width=\textwidth]{../Dissertation/images/part1_img/pattern_structure}
  \end{figure}
  \end{column}
  \begin{column}{0.45\textwidth}
    Углы в БПХ делятся на 4 группы:
    \begin{equation} \notag
    \begin{array}{lll}
    \alpha^\rom{1}_i &= \pi - & \arctan{\frac{N-1-i}{N-1}} \\
    \alpha^\rom{2}_i &= &\arctan{\frac{i - (N-1)}{N-1}} \\
    \alpha^\rom{3}_i &= \frac \pi 2 - & \arctan{\frac{3(N-1)-i}{N-1}} \\
    \alpha^\rom{4}_i &= \frac \pi 2 - & \arctan{\frac{i - 3(N-1)}{N-1}}
    \end{array}
    \end{equation}

    Вычисление прямой проекции в шаге FHT-SIRT имеет вид 
        $\mathrm W = \left( \mathrm  W^\rom{1}\ \mathrm W^\rom{2}\ \mathrm  W^\rom{3}\ \mathrm  W^\rom{4} \right)^{\mathrm T}$.
  \end{column}
\end{columns}
\end{frame}


\begin{frame}
\frametitle{FHT-SIRT}
\framesubtitle{обратная проекция}

\begingroup
\small
\vspace{-0.5cm}
\newtheorem{myth}{Теорема}\
\begin{myth}
Пусть $pattern_j$ --- вертикальный паттерн скоса для j'ой строки преобразования Хафа изображения высотой $M_s = 2^n$.
Тогда имеет место равенство:
\begin{equation} \notag
\label{statement1}
\begin{array}{l l}
pattern_j[i] = pattern_i[j] & \quad  i,j \in \overline{1, 2^n},
\end{array}
\end{equation}
т. е. матрица, составленная из паттернов скоса, записанных в качестве столбцов, симметрична.
\end{myth}
\endgroup
\noindent\rule{8cm}{0.4pt}
\vspace{0.3cm}

Откуда следует, что ${W^{\mathrm K}} ^ {\mathrm T} = W^{\mathrm K}$, а значит
$$
W^{\mathrm T} q = \sum_{K=\rom{1}}^{K=\rom{4}}{W^{\mathrm K} q}
$$

\end{frame}


\begin{frame}
\frametitle{FHT-SIRT}
\framesubtitle{исследование работы}

\begin{columns}[T,onlytextwidth]
  \hspace*{-0.5cm}
  \begin{column}{0.53\textwidth}
    \begin{figure}
      \centering
      \includegraphics[width=\textwidth]{../Dissertation/images/part1_img/time_30_it}
      \caption{Время работы 30 итераций алгоритма}
    \end{figure}
  \end{column}
  \begin{column}{0.6\textwidth}
    \begin{figure}
      \centering
      \includegraphics[width=\textwidth]{../Dissertation/images/part1_img/it_till_stop}
      \caption{количество итераций до останова}
    \end{figure}
  \end{column}
\end{columns}
\end{frame}

\begin{frame}
\frametitle{FHT-SIRT}
\framesubtitle{Предобработка данных}

Перед тем как вести минимизацию с помощью БПХ, необходимо привести измерения в пространство результатов БПХ.
\begin{enumerate}
  \item восстановить соответствие углов проекции строчкам БПХ
  \item растянуть измерения с использованем линейной инерполяции:
\end{enumerate}

\hspace*{2cm}
  \begin{itemize}
    \item \rom{1} растяжение в $\frac 1 {\cos \alpha_i}$ раз
    \item \rom{2} растяжение в $\frac 1 {\cos \alpha_i}$ раз
    \item \rom{3} растяжение в $\frac 1 {\sin \alpha_i}$ раз
    \item \rom{4} растяжение в $\frac 1 {\sin \alpha_i}$ раз
  \end{itemize}

\end{frame}

\begin{frame}
\frametitle{FHT-SIRT}
\framesubtitle{Сходимость при малом количестве углов}
\textbf{Степень разрежения} --- доля отсутствующих углов полного БПХ-пространства
\begin{columns}[T,onlytextwidth]
  \hspace*{-1cm}
  \begin{column}{0.6\textwidth}
  \begin{figure}
    \includegraphics[width=\textwidth]{../Dissertation/images/part1_img/raw}
    \caption{Без регуляризации}
  \end{figure}
  
  \end{column}
  \begin{column}{0.55\textwidth}
    \begin{figure}
    \includegraphics[width=\textwidth]{../Dissertation/images/part1_img/medk}
    \caption{Медианная регуляризация}
    \end{figure}
  \end{column}
\end{columns}
\end{frame}

\begin{frame}
\frametitle{FHT-SIRT}
\framesubtitle{Сходимость при малом количестве углов: регуляризация по Тихонову}

    \begin{figure}
    \includegraphics[width=0.8\textwidth]{tikhonov}
    \end{figure}

\end{frame}

\section{Подавление артефактов, вызванных наличием сильнопоглощающих включений}

\begin{frame}
\frametitle{Модель возникновения артефактов}
  \begin{itemize}[<+->]
    \item при прохождении через сильнопоглощающие включения большая часть излучения поглощается
    \item пиксели детектора имеют некоторый порог активации $\delta_{\mathrm I min}$
    \item если пришедшая интетсивность $\mathrm I_{j} \leq \delta_{\mathrm I min}$, показание детектора будет неотличимо от уровня шума
    \item оптимизационная задача должна учитывать такие измерения особым образом
    \item при логарифмировании условие $\mathrm I_{j} \leq \delta_{\mathrm I min}$ переходит в 
    \begin{equation}
      \label{eq:thresh}
      p_j \geq \delta \left( = \ln \frac {\mathrm I_0}{\delta_{\mathrm I min}}\right)
    \end{equation}
  \end{itemize}
\end{frame}



\begin{frame}
\frametitle{Учет пикселей с высоким поглощением}
Пусь $\mathbb J = \left\{ j | p_j \geq \delta \right\}$, то есть индексы пикселей, для которых выполнено условие (\ref{eq:thresh}).

С учетом пороговой активации, а так же неотрицательности функции $f$, оптимизационная задача будет иметь вид:


\begin{equation} \notag
  % \label{eq:quadprog_ineq}
  \begin{cases}
  \Norm{p - Wf} \rightarrow \min\limits_f & w.r.t \\
  \sum_i f_{i} \omega_{ij} > \delta, & \mbox{если } j \in \mathbb J \\
  f_{i} \geq 0 & \\
  \end{cases}
\end{equation}

\end{frame}


\begin{frame}
\frametitle{Квадратичное программирование}
\begin{itemize}
  \item оптимизация квадратичной функции на наборе линейных ограничений
  \item частный случай выпуклой оптимизации
  \item требует обращения матрицы $W^{\mathrm T} W$. 
  \item для изображения 256х256 и 180 углов при использовании float64 такая матрица будет занимать порядка 32Гб
\end{itemize}


\end{frame}

\begin{frame}
\frametitle{Применение qp для малых размерностей}
\begin{columns}[T,onlytextwidth]
  \hspace*{-1cm}
\begin{column}{0.4\textwidth}
  \begin{figure}
    \centering
    \includegraphics[width=\textwidth]{qp_phantom}
  \end{figure}
  фантом
\end{column}

\begin{column}{0.4\textwidth}
  \begin{figure}
    \centering
    \includegraphics[width=\textwidth]{qp_fbp}
  \end{figure}
  matlab iradon (fpb)
\end{column}

\begin{column}{0.4\textwidth}
    \begin{figure}
    \centering
    \includegraphics[width=\textwidth]{qp_qp_ineq}
  \end{figure}
  QP с ограничениями-неравенствами
\end{column}
\end{columns}

\end{frame}

\begingroup
\small
\begin{frame}
  \frametitle{Как повысить разрешение?}
  \begin{columns}[T, onlytextwidth]
  \hspace{-0.5cm}
  \begin{column}{0.5\textwidth}
    \underline{Метод мягких неравенств} \\ \vspace{0.5cm}
    Введем диагональную матрицу $\mathrm J \in \textup{diag}\{N N_\varphi\}$, такую что 
    $\mathrm J_{j} = \left\{1, \mbox{если}\ j \in \mathbb J, \mbox{иначе }\ 0\right\}$, а так же $\mathrm K = \mathrm E - \mathrm J$.


    \begin{equation} \notag
      \label{eq:soft-ineq}
      \begin{array}{lc}
      \Norm{K(Wf - p)}^2 & + \\
      \alpha \Norm{J(Wf - \delta)}^2  & \to \min\limits_f
      \end{array}
    \end{equation}
  \end{column}

  \begin{column}{0.5\textwidth}
    \underline{Метод барьерных функций} \\ \vspace{0.5cm}
    Заменим неравенства вида $g(x) \leq 0$ на аддитивные барьерные функции $-\frac 1 t \log{-g(x)}$.
    Начиная из ``внутренней''точки, будем минимизировать функционал, постепенно увеличивая t.
    $$
    \begin{array}{ccc}
      F_t(f) = & \Norm{Wf - p}^2 & + \\
              & \frac 1 t \sum \left( -\log{-g(x)} \right) &\to \min\limits_f
    \end{array}{ccc}
    $$

    Так же возможно ввести ослабить ограничения, добавив переменные нежесткости $\xi$

  \end{column}
  \end{columns}
\end{frame}
\endgroup

\begin{frame}
\frametitle{Метод мягких неравенств}
\framesubtitle{фантом}

\begin{figure}
  \centering
  \includegraphics[width=0.9\textwidth]{../Dissertation/images/part2_img/sample}
  \caption{Реконструкции методом мягких ограничений.}
  \label{fig:sample}
\end{figure}

\end{frame}

\begin{frame}
\frametitle{Метод мягких неравенств}
\framesubtitle{реальная реконструкция}

\begin{figure}
\centering
\vspace{-0.6cm}
\begin{tabular}{@{}c@{}c}
    \includegraphics[width=0.50\textwidth]{../Dissertation/images/part2_img/tooth_sino}
&
    \includegraphics[width=0.50\textwidth]{../Dissertation/images/part2_img/soft_ineq_pb_tooth}
\\
   \small a) & \small b)
\end{tabular}
  \caption{a --- Синограмма молочного зуба с включением из свинца. б --- результат восстановления методом мягких ограничений}
\label{fig:tooth_sino_rec}
\end{figure}

\end{frame}

\section{Задача реконструкции при зондировании полихроматическим излучением}




\begin{comment}

\begin{frame}
далее план
\begin{itemize}
\item переходим к предмету исследований, показываем слайд с кружочками и выделяем там предмет исследования
\item рассказываем допущения используемые: параллельная схема, закон Бугера, монохроматичность
\item интегральные vs алгебраические, говорим что рассматривать будем алгебраические.
\item перезодим к целям и задачам, положениям выносимым на защиту
\item далее 1 глава, объясняем про fht, демонстрируем результаты
\item далее 2 глава, объясняем проблему, объясняем модель, постановку задачи квадратичного программирования, результаты, барьерный метод
\item далее переход к полихроматике, 3 глава, градиентный спуск, регуляризация, клиппинг, применение барьерного метода.
\item выводы и discussion.

\end{itemize}
\end{frame}

\end{comment}

\begin{frame}
\frametitle{Результаты работы}
\begin{itemize}
  \item Получено аналитическое выражение для операции транспонированного быстрого преобразования Хафа. Доказана теорема о симметричности четверти БПХ, отвечающей одной группе ориентаций лучей. Получен метод вычислительно эффективного вычисления транспонированного преобразования.
  \item Построен вычислительно эффективный алгебраический метод восстановления измерений компьютерной томографии на основе БПХ. Произведена адаптация метода для работы с данными реальных измерений. Изучены свойства пространства БПХ, влияние различной регуляризации на процесс восстановления.
  \item Построен метод восстановления рентгеновской томографии методом квадратичного программирования для объектов, содержащих сильнопоглощающие включения. На реальных экспериментальных измерениях проведено исследование метода мягких ограничений и сравнение с методом квадратичного программирования.
  \item Построен алгебраический алгоритм восстановления томографии в задаче зондирования полихроматическим излучением, основанный на вычиследнии взвешанных невязок по каждому из элементов, входящих в состав исследуемого образца. Предложены варианты регуляризации для востановления в полихроматической моде. Разработанный метод восстановления позволяет разделить концентрации компонентов смеси при восстановлении.
\end{itemize}
\end{frame}

\section{Спасибо за внимание!}

\end{document} 
