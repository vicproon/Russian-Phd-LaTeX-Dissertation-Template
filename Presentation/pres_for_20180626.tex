\documentclass[12pt]{beamer}
\usepackage[T2A]{fontenc}
\usepackage[utf8]{inputenc}
\usepackage[english,russian]{babel}
\usepackage{mathrsfs,amssymb,amsfonts,amsmath,mathtext}
\usepackage{cite,enumerate,float,indentfirst}
\usepackage{comment}
\usepackage{animate}

\graphicspath{{../images/}{images/}{../Dissertation/images}} 

% \usetheme[secheader]{Boadilla}
% \usecolortheme{seahorse}

\usetheme{Pittsburgh}
\usecolortheme{seahorse}

\beamertemplatenavigationsymbolsempty

\newcommand{\todo}{\alert}
\input{../common/data.tex}      % Основные сведения
\input{../common/newnames.tex} 

\setbeamercolor{footline}{fg=blue}
\setbeamertemplate{footline}{
  \leavevmode%
  \hbox{%
  \begin{beamercolorbox}[wd=.333333\paperwidth,ht=2.25ex,dp=1ex,center]{}%
    % И. О. Фамилия, Организация кратко
    \thesisAuthorShort, \thesisOrganizationShort
  \end{beamercolorbox}%
  \begin{beamercolorbox}[wd=.333333\paperwidth,ht=2.25ex,dp=1ex,center]{}%
    % Город, 20XX
    \thesisCity, \thesisYear
  \end{beamercolorbox}%
  \begin{beamercolorbox}[wd=.333333\paperwidth,ht=2.25ex,dp=1ex,right]{}%
  Стр. \insertframenumber{} из \inserttotalframenumber \hspace*{2ex}
  \end{beamercolorbox}}%
  \vskip0pt%
}
\setbeamertemplate{caption}[numbered]

\newcommand{\itemi}{\item[\checkmark]}

\newcommand{\rom}[1]{%
  \textup{\uppercase\expandafter{\romannumeral#1}}%
}

\newcommand\Wider[2][3em]{%
\makebox[\linewidth][c]{%
  \begin{minipage}{\dimexpr\textwidth+#1\relax}
  \raggedright#2
  \end{minipage}%
  }%
}

\AtBeginSection[]{
  \begin{frame}
  \vfill
  \centering
  \begin{beamercolorbox}[sep=8pt,center,shadow=true,rounded=true]{title}
    \usebeamerfont{title}\insertsectionhead\par%
  \end{beamercolorbox}
  \vfill
      \tableofcontents[currentsection]
  \end{frame}
}

%\title{\small{Название презентации}}
\title{\small{
  АЛГЕБРАИЧЕСКИЙ МЕТОД РЕКОНСТРУКЦИИ В ЗАДАЧЕ КОМПЬЮТЕРНОЙ РЕНТГЕНОВСКОЙ ТОМОГРАФИИ ПРИ ЗОНДИРОВАНИИ МОНО- И ПОЛИХРОМАТИЧЕСКИМ ИЗЛУЧЕНИЕМ
}}

\vspace{60pt}%
\author{\small{%
\ \vspace{30pt} \\
\emph{докладчик:}~\thesisAuthorShort\\%
\emph{научный руководитель:}~\supervisorRegaliaShort~\supervisorFioShort}\\%
%\thesisSpecialtyNumber \ \thesisSpecialtyTitle 
%\\
\vspace{30pt}%
% \thesisOrganization%
\vspace{20pt}%
}
\date{\small{\thesisCity, \thesisYear}}


\begin{document}

\maketitle

\section{Задача восстановления в Компьютерной Томографии}
\begin{frame}
\frametitle{Томографическое измерение}
\textbf{Компьютерная рентгеновская томография} --- 

неразрушающий метод исследования внутренней структуры образцов.
\\ 
Применения:
\begin{itemize}
  \item Контроль качества изготавливаемых изделий
  \item Медицинская диагностика
  \item Использование результатов КТ для дальнейшего моделирования динамических процессов
  \item Биологические и геологические исследования
\end{itemize}

\end{frame}

\begin{frame}
\frametitle{Томографическое измерение}
\centering

\begin{tabular}{p{0.45\textwidth} p{0.55\textwidth}}
  \begin{figure}[H]
    \includegraphics[width=0.45\textwidth]{../Dissertation/images/part1_img/experiment}
  \end{figure}
  &
  \begin{itemize}
  \item $N$ ячеек детектора
  \item $N_\varphi$ углов сканирования
  \item Для каждого угла $\varphi$ и каждой ячейки $\xi$ измеряется интенсивонсть прошедшего рентгеновского излучения \\
    $\mathrm I \left( \varphi, \xi \right) = \mathbb P (f(x, y))$ \\
    $\mathbb P$ --- оператор проекции \\
    $x, y$ --- координаты объекта

  \end{itemize}
\end{tabular}
\end{frame}

\begin{frame}
\frametitle{Томографическое измерение}
\animategraphics[loop,controls,width=\linewidth]{10}{zub_anim/sino_}{000}{39}
\end{frame}

\begingroup
% \small

\begin{frame}
\frametitle{Предположения}
\begin{itemize}
  \item Рассматривается плоское сечение
  \item Параллельная схема проекции
  \item Ослабление интенсивности излучения подчиняется закону Бугера:
\end{itemize}
  $$
  \mathrm I \left( \varphi, \xi , \lambda \right) = \mathrm I_0(\lambda) \exp\left( {-\int_{l(\varphi, \xi)} \! f(l, \lambda) \mathrm d l }\right),
  $$
  \small
  $f(x, y)$ ---  описывает распределение линейного коэффициента ослабления рентгеновского излучения \\
  $\mathrm I(\varphi, \xi, \lambda)$ --- зарегистрированная детектором интенсивность излучения длины волны $\lambda$ \\
  $\mathrm I_0(\lambda)$ --- интенсивность зондирующего излучения \\
  $l(\varphi, \xi)$ --- параметризация прямой под углом $\varphi$, и сдвигом $\xi$ \\
\end{frame}

\begin{frame}
\frametitle{Задача восстановления}
  Связь линейного коэффициента ослабления и интенсивности описывается преобразованием Радона $R[f(x,y)](\varphi, \xi)$:

  $$
  R[f](\varphi, \xi) = 
 \iint \! \mathrm d x \mathrm d y f(x,y)\delta(x\cos\varphi + y\sin\varphi - \xi)
  $$


  После логарифмирования закона ослабления:
  $$
  \ln \left (\frac{\mathrm I_0(\lambda)}{\mathrm I(\varphi, \xi, \lambda)} \right) = p(\varphi, \xi) = R[f](\varphi, \xi)
  $$

\end{frame}

\begin{frame}
\frametitle{Задача восстановления}
\begin{figure}
\centering
    прямая задача, модель измерения\\
    $\rightarrow$
    \\
    \begin{tabular}{p{0.1\textwidth} p{0.8\textwidth} p{01\textwidth}}
    \hspace{-1cm}
    \small{объект}
    &
    \hspace{-1cm} -\includegraphics[width=0.8\textwidth]{sl_sinogram}
    &
    \hspace{-1cm} \small{измерения}
    \\
     \hspace{-1cm} \small{$f(x, y)$}
     &
    &
     \hspace{-1cm} \small{$p(\xi, \varphi)$}
    \end{tabular}
    \\
    $\leftarrow$ \\
    обратная задача, процедура восстановления
\end{figure}
\end{frame}

\begin{frame}
\frametitle{Задача восстановления}

Задача восстановления в компьютерной томографии: \\

по измерениям логарифмов интенсивности пришедшего рентгеновсокго излучения

$$
p(\xi, \varphi),\ \xi = 1 \dots N,\ \varphi = 1 \dots N_\varphi
$$

вычислить функцию распределения линейного коэффициента ослабления 
$$
f(x,y), x = 1 \dots N,\ y = 1 \dots N, 
$$

то есть применить обратное преобразование Радона:

$$
f(x,y) = R^{-1}[p(\varphi, \xi)](x,y)
$$

\end{frame}

\begin{frame}
% \frametitle{Предмет исследования}
Компьютерная томография --- программно-аппаратный комплекс
% \pause
\setlength{\leftmargini}{0em}
\begin{columns}[T,onlytextwidth]
  \begin{column}{.4\textwidth}
  \begin{itemize}
    \item экспериментальная схема
    \item калибровка 
    \item \textbf{программная обработка \ $\rightarrow$}
  \end{itemize}
  \end{column}
  \pause
  \begin{column}[t]{0.4\linewidth}
  \begin{itemize}
    \item предобработка
    \item \textbf{алгоритмы восстановления \ $\rightarrow$}
    \item постобработка
  \end{itemize}
  \end{column}
  \pause
  \begin{column}[t]{0.35\linewidth}
  \begin{itemize}
    \item \textbf{метод оптимизации}
    \item \textbf{модель регуляризации}
    \item \textbf{физическая модель}
    \item реализация математических примитивов
  \end{itemize}
  \end{column}
\end{columns}
\end{frame}
\endgroup

\section{Подходы к решению задачи восстановления в КТ}
\begingroup
\small
\begin{frame}
\frametitle{Методы восстановления}
\begin{tabular}{p{0.15\textwidth} | p{0.4\textwidth} | p{0.4\textwidth}}
\hspace{-1cm} семейство & Интегральные & Алгебраические \\ \hline \vspace{10pt}
\hspace{-1cm} подход & формула для обращения $R^{-1}[p(\varphi, \xi)](x,y)$ & Оптимизационная задача $\min_{f} \Norm{R[f] - p}$\\ \hline \vspace{10pt}
\hspace{-1cm} представители & FBP & ART, SART, SIRT \\ \hline \vspace{10pt}
\hspace{-1cm} сложность & $O(N^2 \log N)$ & $O(N^3)$ \\ \hline \vspace{10pt}
\hspace{-1cm} особенности & универсальность & возможность учета \hspace{1cm} модели объекта или измерительной схемы \\
                          & требует полного набора проекционных углов, & \\ 
                          & их равномерного распределения & \\
                          & чувствительны к шумам & \\
\end{tabular}
\\
\vspace{5pt}
$N$ --- число ячеек детектора
\end{frame}
\endgroup

\begin{frame}
\frametitle{Метод Свертки и обратной проекции}
\begin{columns}
  \hspace{-0.1cm}
  \begin{column}{0.6\textwidth}
    Основные шаги метода:
    \begin{itemize}
    \item Фильтрация проекций: $\tilde{p}(\xi, \varphi) = F^{-1}[|u|F[p(\xi, \varphi)](u)$\\
    $F[\cdot], F^{-1}[\cdot]$ --- прямое и обратное преобразования Фурье по $\xi$\\
    \vspace{10pt}
    \item Вычисление обратной проекции $f(x,y) = \int_0^\pi {\tilde{p} (x \cos\varphi + y \sin \varphi, \varphi) d\varphi}$
    \end{itemize}
  \end{column}

  \begin{column}{0.6\textwidth}
  Достоинства:
  \begin{itemize}
    \item скорость работы
    \item универсальность
    \item точная формула (при непрерывных значениях $\xi, \varphi$)
  \end{itemize}
  Недостатки:
  \begin{itemize}
    \item требует равномерную сетку углов
    \item чувствителен к шумам
    \item не позволяет учитывать специфики эксперимента
  \end{itemize}
  \end{column}
\end{columns}
\end{frame}

\begin{frame}
\centering
\frametitle{Метод Свертки и обратной проекции}
  \includegraphics[width=0.8\textwidth]{fbp_img.png}
  \\
  Операция обратной проекции
\end{frame}

%\section{Вычислительно эффективный алгебраический метод восстановления FHT-SIRT}

\begin{frame}
\frametitle{Алгебраический метод}
\framesubtitle{Основные положения}
\begin{itemize}
  \item непрерывные функции $\rightarrow$ дискретные изображения:

    {
    \centering
    $f(x,y) \rightarrow f_i,\ p(\varphi, s) \rightarrow p_j$
    \par
    }
  \vspace{0.5cm}
  \item преобразование Радона $\rightarrow$ преобразование Хафа:
  
    {
    \centering
    $R[f](\varphi, s) \rightarrow (\mathrm W f)_j$ 
    \par
    }
  \vspace{0.5cm}

    $\mathrm W$ --- матрица проекции, указывает вклад пикселя $i$ в лучевую сумму вдоль луча $j$.\\
    Разреженная матрица размера $N_\varphi * N^3$, в которой только порядка $O(N_\varphi * N^2)$ ненулевых элементов
    \vspace{0.5cm}
  \item решение разреженной СЛАУ большой размерности итерационным методом

    {
    \centering
    $p = \mathrm W f$
    \par
    }

\end{itemize}
\end{frame}

\begin{frame}
\frametitle{Матрица проекции W}
\begin{tabular}{c c c}
\includegraphics[width=0.4\textwidth]{w_matrices/W_10_10_plot.png} &
\includegraphics[height=0.8\textheight]{w_matrices/W_10_35_plot.png} &
\includegraphics[height=0.8\textheight]{w_matrices/W_16_45_plot.png} \\
\small{$N = 10$, $N_\varphi = 10$} &
\small{$N = 10$, $N_\varphi = 35$} & 
\small{$N = 16$, $N_\varphi = 45$}
\end{tabular}
\end{frame}

\begin{frame}
\frametitle{Алгебраический метод}
\framesubtitle{Решение СЛАУ}
\centering
$\Norm{p - \mathrm W f} \rightarrow \min\limits_f$

Оптимизационная задача решается итерационным методом, шаг итерации имеет вид
\vspace{0.5cm}

\begingroup
\footnotesize

\hspace*{-0.5cm}
\begin{tabular}{c|c|c}
ART & SART & SIRT \\ \hline
для каждого луча & для каждого угла & для всех лучей\\
$j = 1 \dots N * N_\varphi$ & $\varphi_k$ & \\
$\hat{f} = f - \gamma \mathrm W^{\mathrm T}_j(\mathrm W f - p)$ &
$\hat{f} = f - \gamma \mathrm {W^{\varphi_k}}^{\mathrm T}(\mathrm W f - p)$ &
$\hat{f} = f - \gamma \mathrm W^{\mathrm T}(\mathrm W f - p)$ \\
\end{tabular}

\vspace{0.5cm}
\raggedright
\endgroup

$\mathrm W_j$ --- столбец матрицы для луча $j$,\\
$\mathrm W^\varphi$ ---  матрица проекции на угол $\varphi$, $\mathrm W = \sum_\varphi {\mathrm W^\varphi}$


\end{frame}

\begin{frame}
\frametitle{SIRT}
\framesubtitle{Анимация процесса восстановления}
\begin{columns}[T,onlytextwidth]
\begin{column}{0.4\textwidth}
\animategraphics[loop,controls,width=\linewidth]{10}{anim/frame_}{1}{60}
\end{column}

\begin{column}{0.6\textwidth}
\animategraphics[loop,controls,width=\linewidth]{10}{anim/slice_}{1}{60}
\end{column}
\end{columns}
\end{frame}

\begin{frame}
\frametitle{Алгебраический метод}

Преимущества:\\
\begin{itemize}
  \item качество восстановления
  \item возможность учета специфики эксперимента (регуляризация, ограничения)
\end{itemize}
\vspace{2cm}

Недостатки:\\
\begin{itemize}
  \item сложность настройки
  \item низкая скорость работы
  \item сходимость
\end{itemize}
\end{frame}

\begin{frame}
\frametitle{Направления развития алгебраических методов}

\begin{itemize}
  \item \textbf{ускорение}
  \begin{itemize}
    \item использование GPU
    \item \textbf{улучшение асимптотики итерации}
    \item ускорение сходимости итерационной процедуры
  \end{itemize}
  \item \textbf{улушчение качества восстановления, борьба с артефактами}
    \begin{itemize}
    \item beam hardening
    \item \textbf{metal artifacts}
    \end{itemize}
  \item \textbf{улучшение возможностей интерпретации результатов восстановления}\\
  \small{В условиях лабороторного спектра восстановленная характеристика является усреднением линейного коэффициента полгощения по спектру}
\end{itemize}

\end{frame}

\section{Вычислительно эффективный алгебраический метод восстановления FHT-SIRT}
% \section{Вычислительно эффективный алгебраический метод восстановления FHT-SIRT}

\begin{frame}
\frametitle{FHT-SIRT: основная идея}
\centering
\begin{columns}
\begin{column}{0.35\textwidth}
\centering
FBP\\
\vspace{20pt}
\includegraphics[width=1\textwidth]{sl_fbp_noisy}\\
\vspace{20pt}
$O(N^2 \log N)$
\end{column}
\vrule{}
\begin{column}{0.35\textwidth}
\centering
SIRT\\
\vspace{20pt}
\includegraphics[width=1\textwidth]{sl_art_good}\\
\vspace{20pt}
$O(N^3)$
\end{column}
\vrule{}
\begin{column}{0.35\textwidth}
\centering
FHT-SIRT\\
\vspace{20pt}
\includegraphics[width=1\textwidth]{sl_art_good}\\
\vspace{20pt}
$O(N^2 \log N)$
\end{column}
\end{columns}
\end{frame}

\begin{frame}
\frametitle{Быстрое преобразование Хафа}
\framesubtitle{БПХ, FHT}
\begin{columns}[T,onlytextwidth]
  \hspace*{-0.5cm}
  \begin{column}{0.65\textwidth}
  Приближенный способ вычисления сумм интенсивностей изображения вдоль всевозможных прямых
  \begin{figure}
    \includegraphics[width=1\textwidth]{fht}
  \end{figure}
  \end{column}
  \begin{column}{0.45\textwidth}
  \begin{itemize}
    \item диадические паттерны суммирования
    \item рекурсивная процедура построения
    \item для больших размеров изображения хорошо приближает прямые (отклонение не превышает  $\frac 1 6 \log N$ ) %\cite{ershov2015dyadic})
    \item асимптотическая сложность $O(N^2 \log N)$
  \end{itemize}
  \end{column}
\end{columns}

\end{frame}

\begin{frame}
\frametitle{Процедура формирования паттернов}
  \begin{figure}
  \centering
    \includegraphics[width=1\textwidth]{../Dissertation/images/part1_img/hough_proc}
  \end{figure}
\end{frame}

\begin{frame}
\frametitle{Быстрое преобразование Хафа}
\framesubtitle{вид матрицы W}
\centering
\begin{columns}

\column{0.2\textwidth}
Обычная проекция \\
$N = 10$ \\
$N_\varphi = 35$

\column{0.3\textwidth}
\includegraphics[height=0.7\textheight]{w_matrices/W_10_35_plot.png}

\column{0.3\textwidth}
\includegraphics[height=0.80\textheight]{w_matrices/W_FHT_10_plot.png}

\column{0.2\textwidth}
БПХ \\
$N = 10$ \\
$N_\varphi = 37$ \\
\end{columns}
\end{frame}

\begin{frame}
\frametitle{Быстрое преобразование Хафа}
\framesubtitle{вид матрицы W}
\centering
\begin{columns}

\column{0.2\textwidth}
\includegraphics[height=0.9\textheight]{w_matrices/W_FHT_8_plot.png}

\column{0.2\textwidth}
$N = 8$ \\
$N_\varphi = 29$

\column{0.1\textwidth}
\includegraphics[height=0.9\textheight]{w_matrices/W_FHT_16_plot.png}

\column{0.2\textwidth}
$N = 16$ \\
$N_\varphi = 61$


\column{0.1\textwidth}
\includegraphics[height=0.9\textheight]{w_matrices/W_FHT_32_plot.png}

\column{0.2\textwidth}
$N = 32$ \\
$N_\varphi = 125$ \\

\end{columns}

\end{frame}

\begin{frame}
\frametitle{FHT-SIRT}
\framesubtitle{прямая проекция}
\begin{columns}[T,onlytextwidth]
  \hspace*{-0.5cm}
  \begin{column}{0.65\textwidth}
  \begin{figure}
    \includegraphics[width=\textwidth]{../Dissertation/images/part1_img/pattern_structure}
  \end{figure}
  \end{column}
  \begin{column}{0.45\textwidth}
    Углы в БПХ делятся на 4 группы:
    \begin{equation} \notag
    \begin{array}{lll}
    \alpha^\rom{1}_i &= \pi - & \arctan{\frac{N-1-i}{N-1}} \\
    \alpha^\rom{2}_i &= &\arctan{\frac{i - (N-1)}{N-1}} \\
    \alpha^\rom{3}_i &= \frac \pi 2 - & \arctan{\frac{3(N-1)-i}{N-1}} \\
    \alpha^\rom{4}_i &= \frac \pi 2 - & \arctan{\frac{i - 3(N-1)}{N-1}}
    \end{array}
    \end{equation}

    Вычисление прямой проекции в шаге FHT-SIRT имеет вид 
        $\mathrm W = \left( \mathrm  W^\rom{1}\ \mathrm W^\rom{2}\ \mathrm  W^\rom{3}\ \mathrm  W^\rom{4} \right)^{\mathrm T}$.
  \end{column}
\end{columns}
\end{frame}


\begin{frame}
\frametitle{FHT-SIRT}
\framesubtitle{обратная проекция}

\begingroup
\small
\vspace{-0.5cm}
\newtheorem{myth}{Лемма}\
\begin{myth}
Пусть $pattern_j$ --- вертикальный паттерн скоса для j'ой строки преобразования Хафа изображения высотой $M_s = 2^n$.
Тогда имеет место равенство:
\begin{equation} \notag
\label{statement1}
\begin{array}{l l}
pattern_j[i] = pattern_i[j] & \quad  i,j \in \overline{1, 2^n},
\end{array}
\end{equation}
т. е. матрица, составленная из паттернов скоса, записанных в качестве столбцов, симметрична.
\end{myth}
\endgroup
\noindent\rule{8cm}{0.4pt}
\vspace{0.3cm}

Откуда следует, что ${W^{\mathrm K}} ^ {\mathrm T} = W^{\mathrm K}$, а значит
$$
W^{\mathrm T} q = \sum_{K=\rom{1}}^{K=\rom{4}}{W^{\mathrm K} q}
$$

\end{frame}


\begin{frame}
\frametitle{FHT-SIRT}
\framesubtitle{исследование работы}

\begin{columns}[T,onlytextwidth]
  \hspace*{-0.5cm}
  \begin{column}{0.53\textwidth}
    \begin{figure}
      \centering
      \includegraphics[width=\textwidth]{fht_sirt_time_30_it}
      \caption{Время работы 30 итераций алгоритма}
    \end{figure}
  \end{column}
  \begin{column}{0.6\textwidth}
    \begin{figure}
      \centering
      \includegraphics[width=\textwidth]{../Dissertation/images/part1_img/it_till_stop}
      \caption{количество итераций до заданного уровня ошибки на разных размерах изображения}
    \end{figure}
  \end{column}
\end{columns}
\end{frame}


% ======================================================
% ================= Сравнение с SART ===================
% ======================================================
\begin{frame}
\frametitle{FHT-SIRT}
\framesubtitle{сравнение с SART}
  \begin{figure}
  \includegraphics[width=0.8\textwidth]{sart__fht_sirt}
  \end{figure}

% \hline
\vspace{0.5cm}


\small
\begin{tabular}{c|c c}
    метод & SART & FHT-SIRT \\ \vspace{5pt}
    количество итераций & 1 & 40 \\ \vspace{5pt}
    время & 0.931c & 0.935c \\ \vspace{5pt}
    ошибка & 11.8845 & 16.1192 \\ \vspace{5pt}
    СКО & 0.3278 & 0.3882 \\ \vspace{5pt}
    N & 256 & 256 \\
\end{tabular}

\end{frame}


\begin{frame}
\frametitle{FHT-SIRT}
\framesubtitle{сравнение с SART: кросс-секции}
\begin{columns}[T,onlytextwidth]
  \hspace*{-1cm}
  \begin{column}{0.4\textwidth}
    \begin{figure}
      \centering
      %\vspace{0.75cm}
      \includegraphics[width=1.5\textwidth]{cs_80_viz}
    \end{figure}
  \end{column}
  \begin{column}{0.6\textwidth}
    \begin{figure}
      \centering
      %\vspace{-1cm}
      \includegraphics[width=1.2\textwidth]{cs_80}
    \end{figure}
  \end{column}
\end{columns}
\end{frame}

\begin{frame}
\frametitle{FHT-SIRT}
\framesubtitle{сравнение с SART: кросс-секции}
\begin{columns}[T,onlytextwidth]
  \hspace*{-1cm}
  \begin{column}{0.4\textwidth}
    \begin{figure}
      \centering
      %\vspace{0.75cm}
      \includegraphics[width=1.5\textwidth]{cs_127_viz}
    \end{figure}
  \end{column}
  \begin{column}{0.6\textwidth}
    \begin{figure}
      \centering
      %\vspace{1cm}
      \includegraphics[width=1.2\textwidth]{cs_127}
    \end{figure}
  \end{column}
\end{columns}
\end{frame}


\begin{frame}
\frametitle{FHT-SIRT}
\framesubtitle{сравнение с SART: кросс-секции}
\begin{columns}[T,onlytextwidth]
  \hspace*{-1cm}
  \begin{column}{0.4\textwidth}
    \begin{figure}
      \centering
      %\vspace{0.75cm}
      \includegraphics[width=1.5\textwidth]{cs_v_50_viz}
    \end{figure}
  \end{column}
  \begin{column}{0.6\textwidth}
    \begin{figure}
      \centering
      %\vspace{-1cm}
      \includegraphics[width=1.2\textwidth]{cs_v_50}
    \end{figure}
  \end{column}
\end{columns}
\end{frame}


\begin{frame}
\frametitle{Выводы}
\begin{itemize}
  \item Построен асимптотически эффективный алгоритм вычисления обратной проекций с помощью быстрого преобразования Хафа (БПХ)
  \item Использование БПХ позволяет снизить асимпотику итерации метода SIRT с $O(N^3)$ до $O(N^2 \log N)$
  \item Построенный алгоритм FHT-SIRT позволяет добиться качества восстановления, аналогичного качеству традиционных алгоритмов
\end{itemize}
\end{frame}

\section{Подавление артефактов, вызванных наличием сильнопоглощающих включений}
% \section{Подавление артефактов, вызванных наличием сильнопоглощающих включений}

\begin{frame}
\frametitle{Примеры артефактов}

\begin{columns}[t]
\column{.5\textwidth}
\centering
\includegraphics[height=0.35\textheight]{../Dissertation/images/part2_img/tooth_artifacts_med}\\ \vspace{5pt}
\includegraphics[height=0.55\textheight]{20180531_tooth_pb_slice}
\column{.5\textwidth}
\centering
\includegraphics[height=0.35\textheight]{../Dissertation/images/part2_img/tooth_artifacts_med_3}\\ \vspace{5pt}
\includegraphics[height=0.55\textheight]{20180531_tooth_pb_vol}
\end{columns}

\end{frame}

\begin{frame}
\frametitle{Модель возникновения артефактов}
  \begin{itemize}[<+->]
%    \item при прохождении через сильнопоглощающие включения большая часть излучения поглощается
    \item пиксели детектора имеют некоторый порог активации $\delta_{\mathrm I min}$
    \item если пришедшая интетсивность $\mathrm I_{j} \leq \delta_{\mathrm I min}$, показание детектора будет неотличимо от уровня шума
%    \item при логарифмировании условие $\mathrm I_{j} \leq \delta_{\mathrm I min}$ переходит в 
    \begin{equation}
      \label{eq:thresh}
      p_j \geq \delta \left( = \ln \frac {\mathrm I_0}{\delta_{\mathrm I min}}\right)
    \end{equation}

    \item оптимизационная задача должна учитывать такие измерения особым образом
  \end{itemize}
\end{frame}



\begin{frame}
\frametitle{Учет пикселей с высоким поглощением}
Пусть $\mathbb J = \left\{ j | p_j \geq \delta \right\}$, то есть индексы пикселей, для которых выполнено условие (\ref{eq:thresh}).

С учетом пороговой активации, а так же неотрицательности функции $f$, оптимизационная задача будет иметь вид:


\begin{equation} \notag
  % \label{eq:quadprog_ineq}
  \begin{cases}
  \Norm{p - Wf} \rightarrow \min\limits_f & w.r.t \\
  \sum_i f_{i} \omega_{ij} > \delta, & \mbox{если } j \in \mathbb J \\
  f_{i} \geq 0 & \\
  \end{cases}
\end{equation}

\end{frame}


\begin{frame}
\frametitle{Квадратичное программирование}
\begin{itemize}
  \item оптимизация квадратичной функции на наборе линейных ограничений
  % \item частный случай выпуклой оптимизации
  \item требует обращения матрицы $W^{\mathrm T} W$. 
  \item для изображения 256х256 и 180 углов при использовании float64 такая матрица будет занимать порядка 12Гб
\end{itemize}

\end{frame}

\begin{frame}
\frametitle{Применение qp для малых размерностей}
\begin{columns}[T,onlytextwidth]
  \hspace*{-1cm}
\begin{column}{0.4\textwidth}
  \begin{figure}
    \centering
    \includegraphics[width=\textwidth]{qp_phantom}
  \end{figure}
  фантом
\end{column}

\begin{column}{0.4\textwidth}
  \begin{figure}
    \centering
    \includegraphics[width=\textwidth]{qp_fbp}
  \end{figure}
  matlab iradon (fpb)
\end{column}

\begin{column}{0.4\textwidth}
    \begin{figure}
    \centering
    \includegraphics[width=\textwidth]{qp_qp_ineq}
  \end{figure}
  QP с ограничениями-неравенствами
\end{column}
\end{columns}

\end{frame}

\begingroup
\small
\begin{frame}
  \frametitle{Как повысить разрешение?}
  \begin{columns}[T, onlytextwidth]
  \hspace{-0.5cm}
  \begin{column}{0.5\textwidth}
    \underline{Метод мягких неравенств} \\ \vspace{0.5cm}
    Введем диагональную матрицу $\mathrm J \in \textup{diag}\{N N_\varphi\}$, такую что 
    $\mathrm J_{j} = \left\{1, \mbox{если}\ j \in \mathbb J, \mbox{иначе }\ 0\right\}$, а так же $\mathrm K = \mathrm E - \mathrm J$.


    \begin{equation} \notag
      \label{eq:soft-ineq}
      \begin{array}{lc}
      \Norm{K(Wf - p)}^2 & + \\
      \alpha \Norm{J(Wf - \delta)_{+}}^2  & \to \min\limits_f
      \end{array}
    \end{equation}
  \end{column}

  \begin{column}{0.5\textwidth}
    \underline{Метод барьерных функций} \\ \vspace{0.5cm}
    Заменим неравенства вида $g(x) \leq 0$ на аддитивные барьерные функции $-\frac 1 t \log{\left(-g(x)\right)}$.
    Начиная из ``внутренней''точки, будем минимизировать функционал, постепенно увеличивая t.
    $$
    \begin{array}{ccc}
      F_t(f) = & \Norm{Wf - p}^2 & + \\
     \frac 1 t \sum \left( -\log{\left(-g(x)\right)} \right) &\to \min\limits_f &
    \end{array}{ccc}
    $$

    Так же возможно ослабить ограничения, добавив переменные нежесткости $\xi$

  \end{column}
  \end{columns}
\end{frame}
\endgroup

\begin{frame}
\frametitle{Резульататы численного эксперимента}
\framesubtitle{к изображениям применена гамма-коррекция}

\begin{figure}
  \centering
  \vspace{-0.3cm}
  \includegraphics[height=0.7\textheight]{qp_foursome}
  \caption{сравнение реконструкции различными методами}
  \label{fig:sample}
\end{figure}

\end{frame}

\begin{frame}
\frametitle{Резульататы численного эксперимента}
\framesubtitle{к изоюражениям применена гамма-коррекция}

\begin{figure}
  \centering
  % \vspace{-0.3cm}
  \includegraphics[width=\textwidth]{qp_cs_x28}
  \caption{кросс-секции реконструкций}
  \label{fig:sample}
\end{figure}

\end{frame}


\begin{frame}
\frametitle{Метод мягких неравенств}
\framesubtitle{реконструкция экспериментальных данных}

\centering
\vspace{-0.3cm}
\begin{columns}
\vspace{-1.2cm}
\begin{column}{0.3\textwidth}
\begin{figure}
    \includegraphics[width=1\textwidth]{zub_photo}
    \caption{Образец: молочный зуб с включением из свинца}
\end{figure}
\end{column}

\begin{column}{0.85\textwidth}
\vspace{-0.3cm}
\begin{figure}
    \centering
    \includegraphics[width=\textwidth]{../Dissertation/images/part2_img/pb_big__fbp_vs_soft__cs__viridis} \\

    \caption{Результаты восстановления}
    \label{fig:fbp_vs_soft__zub}
\end{figure}
\end{column}
\end{columns}

\end{frame}

\begin{frame}
\frametitle{Выводы}
\begin{itemize}
  \item Подход, основанный на ограничениях-неравенствах, позволяет улучшить качество восстановления в рамках предложенной модели
  \item На модельных данных метод барьерных функций показывает лучшее, чем метод мягких ограничений или метод FBP, качество восстановления
  \item На реальных экспериментальных измерениях показано, что метод мягких ограничений позволяет добиться локально-постоянной интенсивности сильнопоглощающего включения
  \item Публикации \cite{ecms2015Chukalina, icmv2015Chukalina,trudi_isa_2018}
\end{itemize}
\end{frame}

\section{Задача реконструкции при зондировании полихроматическим излучением}
%\section{Задача реконструкции при зондировании полихроматическим излучением}

\begingroup
\small
\begin{frame}
\frametitle{Модель измерений}

При переходе к немонохроматическому случаю, уравнение затухания прошедшей через объект интенсивности: 
\begin{equation}
\notag
I(\varphi, \xi) = \int_0^{+\infty}{\left\{
  I_0(\lambda) \exp{\left(- {\mathrm R}[f(\lambda)](\varphi, \xi) \right)} d\lambda
  \right\}}  
\end{equation}


Будем считать, что объект состоит из $K$ элементов, с неизвестными пространственными концентрациями $c_k(x,y)$, и известными табулированными массовыми коэффициентами ослабления $\kappa_k(\lambda)$.
Тогда
$$
f(x,y, \lambda) = \sum_{k = 1} ^K {c_k(x,y) * \kappa_k(\lambda)}
$$

и прямая проекция для пикселя $j$ принимает вид

\begin{equation} \notag
  \label{eq:white_fp_final}
  I(c)_j = \int_0^{+\infty} {d\lambda \left\{
    I_0(\lambda) \exp{\left(
      -\sum_{k=1}^K {\rho \kappa_k(\lambda) (W c_k)_j} 
      \right)}
  \right\}}
\end{equation}

\end{frame}
\endgroup

\begin{frame}
\frametitle{Оптимизационная задача}

Следуя алгебраическому подходу, реконструкция проводится с помощью градиентного спуска оптимизации квадратичной функции потерь 
$$
Q(c) = \frac {\left(I(c) - t\right)^2} {S} \to \min \limits_c,
$$
где $t$ --- измеренные интенсивности прошедшего через объект излучения, \\
$S = \int_0^{+\infty} { I_0(\lambda) d\lambda}$
 --- суммарная интенсивность зондирующего излучения.

\end{frame}

\begin{frame}
\frametitle{Градиент функции потерь}
\begin{equation} \notag
\label{eq:part3_whitegrad}
  \nabla_k \ Q = 2W^\intercal R_k \text{, где } R_{kj} = \frac {(I(c) - t)_j} {S} \mu_{kj}
\end{equation}

а формулы для вычисления весов невязок по каждому элементу приведены ниже:
\begin{equation} \notag
  \label{eq:weights}
  \mu_{k} = \int_0^{+\infty} {d\lambda \left\{
    -\rho \kappa_k(\lambda) 
    I_0(\lambda)
    \exp{\left(
      -\sum_{s=1}^K {\rho \kappa_s(\lambda) (W c_s)} 
         \right)}
    \right\}}
\end{equation}

\end{frame}


\begin{frame}
\frametitle{Метод взвешенных невязок}
\framesubtitle{регуляризация}
  \begin{block}{Мультипликативная регуляризация}
    Запрещаем одновременное нахождение разных элементов в одном пикселе, т.е. $c_{k} \odot c_{s} = 0 \mbox{, если} k \neq s$.

    $$
    Q(c) + \beta\sum_{k_1 != k_2}\Norm{c_{k_1} \odot c_{k_2}} \to \min \limits_c
    $$
  \end{block}
  \begin{block}{Ограничение на значения концентраций}
  Используя метод барьерных функций, можем учесть, что концентрации имеют значения на интервале $[0, 1]$
    $$
    \begin{array}{lc}
    Q(c) \to \min \limits_c & w.r.t \\
    c_k \geq 0 & \\
    c_k \leq 1
    \end{array}
    $$
  \end{block}
\end{frame}

\begin{frame}
\frametitle{Исходные данные}
\begin{figure}
\centering
\includegraphics[width=\textwidth]{0999}
\\
\caption{фантом на тривиальных спектрах}
\end{figure}
\end{frame}


\begin{frame}
\frametitle{Исходные данные}
\begin{figure}

\centering
\includegraphics[height=0.7\textheight]{../Dissertation/images/part3_img/synth_spectre}
\\
\caption{спектры и массовые коэффициенты ослабления элементов}
\end{figure}

\end{frame}

\begin{frame}
\frametitle{Результаты восстановления}
\begin{figure}
\centering
\includegraphics[width=\textwidth]{whiterec_res}
\\
\caption{Метод взвешенных невязок + метод барьерных функций + мультипликативная регуляризация}
\end{figure}
\end{frame}

\begin{frame}
\frametitle{Реальные экспериметнальные данные}
\centering
\vspace{-0.3cm}
\begin{columns}

\begin{column}{0.2\textwidth}
\begin{figure}
    \includegraphics[width=1\textwidth]{zub_photo}
    \caption{зуб (Ca) с включением из свинца (Pb)}
\end{figure}
\end{column}

\begin{column}{0.8\textwidth}
\begin{figure}
    \includegraphics[width=1.1\textwidth]{zub_spectre}
\end{figure}
\end{column}
\end{columns}
\end{frame}

\begin{frame}
\frametitle{Реальные экспериметнальные данные}
\includegraphics[width=\textwidth]{zub_wrm}
\end{frame}

\begin{frame}
\frametitle{Выводы}
\begin{itemize}
  \item Выведен шаг итерации в задаче, поставленной относительно концентраций
  \item Приведены результаты реконструкции на модельных и реальных экспериментальных данных
  \item Несмотря на то что методу не удается разделить концентрации, удается получить полезные свойства восстановленных картин
  \item Публикации \cite{cryst-2018, itas2015Prun}
\end{itemize}
\end{frame}


\begin{frame}
\frametitle{Заключение}
\begin{itemize}
  \item Рассмотрена задача восстановления в компьютерной томографии, основные подходы к ее решению.
  \item Построен асимптотически быстрый алгебраический метод FHT-SIRT, основанный на быстром преобразовании Хафа.
  \item Представлен метод восстановления, позволяющий уменьшить негативное влияние сильнопоглощающих включений в объекте.
  \item Предложен метод восстановления относительно концентраций различных элементов в объекте.
\end{itemize}
\end{frame}

\begin{frame}
\centering
%\vfill(5cm)
\large {Спасибо за внимание!}
\end{frame}

\begin{frame}
\frametitle{Приложение 1}
\framesubtitle{Полученные результаты}
\begin{itemize}
  \item Получен метод вычислительно эффективного вычисления транспонированного преобразования БПХ.
  \item Построен вычислительно эффективный алгебраический метод восстановления измерений компьютерной томографии на основе БПХ.
  \item Произведена адаптация метода для работы с данными реальных измерений. \item Произведено сравнение нового метода FHT-SIRT с традиционным SART
\end{itemize}
\end{frame}

\begin{frame}
\frametitle{Приложение 1}
\framesubtitle{Полученные результаты}
\begin{itemize}
  \item Построен метод восстановления рентгеновской томографии для объектов, содержащих сильнопоглощающие включения. 
  \item На реальных экспериментальных измерениях проведено исследование метода мягких ограничений, произведено сравнение с методом квадратичного программирования.
  \item Построен алгоритм восстановления томографии в задаче зондирования полихроматическим излучением. 
  \item Продемонстрированы результаты восстановления как синтетических, так и реальных экспериментальных данных
\end{itemize}
\end{frame}

\end{document} 

