%% Согласно ГОСТ Р 7.0.11-2011:
%% 5.3.3 В заключении диссертации излагают итоги выполненного исследования, рекомендации, перспективы дальнейшей разработки темы.
%% 9.2.3 В заключении автореферата диссертации излагают итоги данного исследования, рекомендации и перспективы дальнейшей разработки темы.
\begin{enumerate}
  \item Получено аналитическое выражение для операции транспонированного быстрого преобразования Хафа. Доказана теорема о симметричности четверти БПХ, отвечающей одной группе ориентаций лучей. Получен метод вычислительно эффективного вычисления транспонированного преобразования.
  \item Построен вычислительно эффективный алгебраический метод восстановления измерений компьютерной томографии на основе БПХ. Произведена адаптация метода для работы с данными реальных измерений. Изучены свойства пространства БПХ, влияние различной регуляризации на процесс восстановления.
  \item Построен метод восстановления рентгеновской томографии методом квадратичного программирования для объектов, содержащих сильнопоглощающие включения. На реальных экспериментальных измерениях проведено исследование метода мягких ограничений \todo{и сравнение с методом квадратичного программирования}
  \item Построен алгебраический алгоритм восстановления томографии в задаче зондирования полихроматическим излучением, основанный на вычиследнии взвешанных невязок по каждому из элементов, входящих в состав исследуемого образца. Предложены варианты регуляризации для востановления в полихроматической моде. \todo{описать, какой получен результат}
\end{enumerate}
