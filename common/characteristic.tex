{\actualityandprogress}

Компьютерная рентгеновская томография --- это неразрущающий метод исследования внутренней структуры объектов.
Он позволяет по набору измеренных под разными углами сканирования рентгеновских проекций получить пространственное распределение коэффициента поглощения излучения.
Это распределение в дальнейшем используется для контроля качества в промышленности, постановки медицинских диагнозов, моделировании процессов в объекте, поэтому ошибки в полученных данных могут катастрофически повлиять на прикладной результат измерений.

С математической точки зрения задача восстановления измерений в комьютерной томографии эквивалентна обращению преорбразования радона пространственной функции поглощения. 
Классические алгоритмы были разработаны и освещены в классических работах зарубежных ученых Кормака, Хаунсфилда, Гордона и Андерсена, а так же в работах советских ученых Вайнштейна, Орлова.
Выделяют две группы численных методов решения задачи реконструкции: интегральные (аналитические) и алгебраические (итерационные) методы.
Основными преимуществами интегральных методов является быстродействие и универсальность, в то время как алгебраические позволяют добиться лучшего качества восстановления за счет учета специфики математической модели процесса измерения.

Будучи сложным программно-аппаратным комплексом, компьютерный томограф содержит множество потенциальных источников ошибки.
Однако даже если экспериментальная схема настроена и откалибрована идеально, ошибки могут возникать из-за неправильной работы алгоритмов восстановления, выполняющих преобразование измеренных проекционных данных в искомое пространтсвенное распределение.
Это подтверждается возникновением характерных ошибок или артефактов восстановление и при компьютерном моделировании процесса измерения, когда неточности постановки физического экспериметна исключены полностью, или их параметры заданы и точно известны.

Важный класс возникающих при реконстркции артефактов связан с наличием в исследуемом объекте материалов, обладающих существенно разными поглощающими способностями.
Это и так называемые металлические артефакты, вызванные наличием сильнопоглощаюзих включений, и артефакты, возникающие при зондировании объекта полихроматическим излучением.
В последнем случае восстанавливаемая характеристика вообще является некоторой усредненной плотностью коэффициента поглощения и не имеет единого физического смысла, а может быть использована только для семантической интерпретации экспертом.
Описанные недостатки можно устранять как на уровне экспериментальной схемы, так и на уровне алгоритмов реконструкции.
Много современных исследований посвящено схемам со сканированием на разных энергиях, а так же использовании энергодисперсионных детекторов.
Недостатком первого подхода является увеличение дозы излучения на исследуемый объект, что существенно для медицинских приложений.
Недостатком второго подхода является малое количество фотонов приходящее в отдельный канал, и, как следствие, низкое соотношение сигнал / шум. 
Это влечет необходимость в дополнительной модификации алгоритмов восстановления для достижения требуемого качества реконструкции.

Необходимость сканирования на низких дозах энергии, а так же учета специфики экспериментальной схемы, естественным образом предполагает реконструкцию с помощью алгебраических методов.
В настоящее время развитием методов восстановления КТ занимаются такие зарубежные ученые, как Э. Сидки, Й. Жанг, Ф. Мейер, из отечественных --- Д. Казанцев, В. Пикалов, О. Филонин.
В частности перспективными являются подходы, формулирующие специфические свойства измерительного эксперимента в виде аддитивных штрафов к целевой функции (регуляризация) и ограничений-неравенств на результаты восстановления.
Данные подходы являются применимыми к задаче восстановления КТ в алгебраической постановке ввиду ее выпуклости, однако требуют разработки новых комплексов программ, реализующих алгоритмы, ввиду крайне высокой размерности решаемой задачи.



% классики зарубежные: А.Кормак, Г. Хаунсфилд, Р. Гордон, А.Х. Андерсен. (+)

% классики советские: БК Вайнштейн, ДИ Орлов (+)

% современные зарубежные: Э. Сидкий, Х. Жанг, Е. Мейер, 

% современные советские: ОВ Филонин, ВВ Пикалов, ДИ Казанцев


{\aim} ~данной работы являются разработка метода реконструкции, позволяющего учесть присутствие в объекте сильнопоглощающих включений, а так же метода численной интерпретации результатов измерений многокомпонентных объектов.

Для достижения поставленной цели были решены следующие {\tasks}:
\begin{enumerate}
  \item построен асимптотически быстрый алгебраический метод реконструкции, основанный на применении быстрого преобразования Хафа.
  \item доказана сходимость построеного алгебраического метода реконструкции, за счет полученного математического выражения градиента быстрого преобразования Хафа.
  \item построен алгоритм реконструкции для объектов, содержащих сильнопоглощающие включения.
  \item построен алгоритм реконструкции, учитывающий покомпонентное ослабление полихроматического спектра.
\end{enumerate}

{\novelty}
\begin{enumerate}
  \item Впервые для реконструкции томографических измерений было применено быстрое преобразование Хафа.
  \item Впервые получено выражение для производной быстрого преобразования Хафа, а так же алгоритм его эффективного вычисления.
  \item Построен алгоритм реконструкции, учитывающий вклад сильно поглощающих включений с помощью оригинальной модели ограничений-неравенств.
  \item Предложена схема обработки данных полихроматического зондирования, при которой восстанавливаются реальные физические концентрации элементов.
\end{enumerate}

{\influence} ~Результаты, полученные в диссертационной работе, используются для обработки данных лабораторных исследований. Построенные алгоритмы лягут в основу программного обеспечения новых моделей промышленных томографов.

Полученное в работе выражение для градиента быстрого преобразования Хафа имеет общетеоретическое значение и уже применяется в области машинного обучения для обратного распространения ошибки в нейронных сетях глубокого обучения через слой БПХ.

{\methods}
Для решения задач реконструкции томографических измерений используются методы теории условной и безусловной оптимизации: градиентные методы оптимизации, квадратичное программирование, регуляризация.
Для ускорения итерации алгебраического метода используются алгоритмы обработки изображений в виде быстрого преобразования Хафа.


{\defpositions}
\begin{enumerate}
  \item Предложен эффективный вычислительный метод решения задачи томографической реконструкции FHT-SIRT, основанный на алгебраическом подходе, который позволяет снизить асимптотическую оценку сложности вычисления итерации с $O(n^3)$ до $O(n^2~\log n)$, что подтвержается численными экспериментами и замерами времени работы программной реализации алгоритма.
  \item Проведено математическое обоснование сходимости предложенного метода.
  \item Предложен метод реконструкции на основе квадратичного программирования, который позволяет уменьшить артефакты на восстановленном изображении, вызванные наличием сильно поглощающих областей в зондируемом объекте.
  \item Предложен алгебраический метод реконструкции для случая полихроматического зондирования, который решает оптимизационную задачу реконструкции относительно линейной комбинации концентраций с ограничениями-неравенствами на их область значений.
\end{enumerate}


{\reliability} полученных результатов обеспечивается модельными экспериментами и численными симуляциями, а так же экспериментами с восстановлением реально измеренных в лабораторных условиях образцов.\ Результаты находятся в соответствии с результатами, полученными другими авторами.


{\probation}
Основные результаты работы докладывались~на: конферециях 
35-я конференция молодых ученых и специалистов «Информационные технологии и системы» (2012, 19 - 25 августа, Петрозаводск, Россия),
11th Biennal Conference on High Resolution X-Ray Diffraction and Imaging (XTOP 2012, St. Petersburg, Russia), 
29th European Conference on Modelling and Simulation (ECMS 2015, Albena, Bulgaria),
Eighth International Conference on Machine Vision (ICMV 2015, Barcelona, Spain),
на общефизическом семинаре ИПТМ РАН (октябрь 2016).

{\contribution} Все результаты диссертации, вынесенные на защиту, получены автором самостоятельно.
Автором самостоятельно реализованы методы восстановления FHT-SIRT из первой главы, барьерных функций из второй, метод взвешанных невязок из третьей, проведены численные экспериметны по обработке реальных и модельных данных.
Постановка задач и обсуждение результатов проводились совместно с научным руководителем.
Генерация модельных данных для экспериментов в полихроматике проводилась аспирантом факультета КН НИУ ВШЭ Ингачевой А.~С. 
Программная имплементация метода мягких ограничений, использованная для сравнения с методом барьерных функций во второй главе, принадлежит Соколову В.~В.
Измерения для экспериментов по восстановлению зуба со свинцовым включением производились на лабоработном источнике ИК РАН в лаборатории рефлектометрии и малоуглового рассеяния.
Многие аспекты исследований в разное время обсуждались с Чукалиной М.~В., Николаевым Д.~П., Бузмаковым А.~В., Ингачевой А.~С., Соколовым В.~В.


\ifthenelse{\equal{\thebibliosel}{0}}{% Встроенная реализация с загрузкой файла через движок bibtex8
    \publications\ Основные результаты по теме диссертации изложены в 11 печатных изданиях, 
    3 из которых изданы в журналах, рекомендованных ВАК, 
    6 "--- в тезисах докладов.%
}{% Реализация пакетом biblatex через движок biber
%Сделана отдельная секция, чтобы не отображались в списке цитированных материалов
  \begin{refsection}%
    \printbibliography[heading=countauthorvak, env=countauthorvak, keyword=biblioauthorvak, section=1]%
    \printbibliography[heading=countauthornotvak, env=countauthornotvak, keyword=biblioauthornotvak, section=1]%
    \printbibliography[heading=countauthorconf, env=countauthorconf, keyword=biblioauthorconf, section=1]%
    \printbibliography[heading=countauthor, env=countauthor, keyword=biblioauthor, section=1]%

    \publications\ Основные результаты по теме диссертации изложены в \arabic{citeauthor} печатных изданиях \nocite{PruBuzNik13, Prun2013Crys, Vestnik2016, Prun2013AutomAndRemCont, buz_jac_2015, chukalina2014xray}, 
    \arabic{citeauthorvak} из которых изданы в журналах, рекомендованных ВАК %\nocite{PruBuzNik13, Prun2013Crys, Vestnik2016}, 
    \arabic{citeauthorconf} "--- в тезисах докладов \nocite{itas2015Prun,itas2015Ingacheva,ecms2015Chukalina, icmv2015Chukalina, embc2013Buzmakov, nikolaevfast}.
  \end{refsection}
}