% \begin{comment}
%Для успешного решения многих прикладных задач необходимой является возможность подробного исследования внутренней структуры объектов без их физического изменения или разрушения.
%Речь идет о таких областях применения как медицина, промышленность, биология, геология и др. 
%Методы, позволяющие осуществлять такие исследования, назыавются томографическими.


%\begin{comment}
%Современное развитие вычислительной техники, повышение ее доступности, появление новых инструментариев обработки информации, развитие алгоритмов и компьютерных наук в целом, делают возможным применение алгоритмов обработки изображений для анализа результатов томографических измерений.
%%\end{comment}
%В данной работе рассматриваются методы обработки измерений в рентгеновской томографии.
%Исследуемый объект зондируется рентгеновским излучением под разными углами, называемыми проекционными.
%Данные томографической проекции, т.е. измеренное ослабление рентгеновского излучения, позволяют восстановить внутренние характеристики объекта, а именно пространственное распределение коэффициента ослабления рентгеновского излучения.

%Центральную роль в подобных измерениях играет томограф --- программно-аппаратный комплекс, в котором совмещены как измерительная часть, так и математическая обработка полученных данных.
%Аппаратная часть состоит из источника, держателя образца, оптики и детектора, регистрирующего прошедшее излучение. 
%Конфигурации измерительной схемы могут меняться по целому ряду параметров. 
%Так, могут меняться форма зондирующего пучка, его спектр, спектр чувствительности детектора и его тип, взаимное расположение детектора, источника и держателя образца.
%Зачастую, например, в медицинских измерениях, критическую роль играет еще и доза облучения, поглащаемая объектом, что влечет за собой изменения в мощности источника, времени измерения и динамике исследуемого объекта.

%``Сырые'' данные, полученные непосредственно после измерений, не пригодны для их визуального анализа экспертом.
%Для того чтобы получить из измеренных данных, так же называемых синограммой, искомые пространственные распределения характеристик объектов, необходимо произвести дополнительную обработку, называемую методами восстановления компьютерной томографии (КТ).
%Программную часть томографа составляют алгоритмы восстановления томографических измерений, а так же управления механическими частями томографа, детектором и излучателем.
%Эта диссертационная работа исследует проблемы, возникающие именно в алгоритмах восстановления, т.е. в программной части томографа.

% \end{comment}

{\actualityandprogress} 

Исследование внутренней структуры объектов без их физического изменения или разрушения необходимо для решения прикладных задач во многих областях:  медицина, промышленность, биология, геология.
Методы, позволяющие осуществлять такие исследования, называются томографическими.
В данной работе рассматриваются методы обработки результатов измерений рентгеновских томографов.
Исследуемый объект зондируется рентгеновским излучением под разными углами, называемыми проекционными, и интенсивность прошедшего излучения измеряется позиционно-чувствительным детектором.
Измеренные данные, считанные с детектора, при некоторых условиях после логарифмирования представляют собой преобразование Радона функции пространственного распределения коэффициента ослабления излучения.
Для прикладных исследований интерес представляет именно исходное распределение, то есть обратное преобразование Радона от измеренных данных.
Задача компьютерной томографии состоит в обращении преобразования Радона и получении исходного распределения.
В условиях конечного числа проекционных углов, ограниченного разрешения детектора, наличия шумов в измерениях, неточностях в экспериментальной схеме, а так же ограниченной точности вычислительных схем, задача восстановления проекционных данных не может быть решена точно.
При этом наличие неточностей в восстановленной характеристике (артефактов), может служить причиной серьезных ошибок в дальнейших исследованиях: в медицине это может быть неправильно поставленный диагноз, в промышленности - некорректные численные рассчеты моделей, в прототипировании - непригодность изготовленных с помощью 3D-печати моделей.
Поэтому важной задачей является разработка алгоритмов, минимизирующих ошибки восстановления и артефакты.

Обычно в промышленных томографах для задачи обращения преобразования Радона используется алгоритм свертки и обратной проекции (Filtered back projection, FBP).
Этот алгоритм сочетает в себе быструю скорость работы и приемлемое качество восстановления на широком спектре объектов.
Однако зачастую, чтобы добиться лучшего качества восстановления, необходимо применять более ресурсоемкие итерационные алгебраические методы восстановления.
Это позволяет учитывать специфику решаемой задачи, как за счет использования регуляризации в процессе восстановления, так и за счет учета специфики задачи в моделировании построения проекций.
Алгебраические методы незаменимы при импользовании измерительных схем с малым количеством проекционных углов. 
Поэтому особенно важными для применения алгебраических методов являются их эффективные реализации, а так же исследование возможностей их модификации при восстановлении реальных данных.
Необходимой является разработка алгоритмов, способных избежать появления характерных ошибок восстановления: при наличии металлических включений, при зондировании полихроматическим излучением.

\begin{comment}
~Метод рентгеновской томографии применятеся для исследования объектов, которые могут различаться по широкому списку параметров: размер, химический состав, динамика изменения внутренней структуры, требования к мощности излучения и др.
Наличие ошибок в восстановленных данных может крайне негативно сказаться на дальнейшем использовании результатов в приложениях.
Например, наличие ложного пика в восстановленной характеристике может быть неверно интерпретировано экспертом-медиком как опухоль.
Из-за наличия шумов в восстановленной модели ее дальнейшее использование для, например, прототипирования с использованием 3D-печати может оказаться невозможным.
Наконец, отсутствие внутренней особенности на восстановленной картине может коренным образом изменить результаты численного моделирования физических процессов в изучаемом образце.
Таким образом, однин из важнейших задач в разработке алгоритмов восстановления измерений компьютерной томографии --- анализ типов, причин возникновения ошибок восстановления и борьба с ними.

Современные промышленные томографы, как правило, комплектуются программным обеспечением компьютерной томографии общего назначения.
В ``ядре'' такого ПО, как правило, используется алгоритм восстановления из семейства методов свертки и обратной проекции (Filtered Backprojection, FBP) 
%\cite{commercialCTemployFBP},
 с возможностью выбора пред- и постобработки.
% В зависимости от области применения, в экспериментальной схеме могут меняться конструкция томографа, спектр зондирующего излучения, точность механических подвижек, форма восстанавливающего луча, а так же физические свойста самого объекта исследования.
Обычно ПО, обрабатывающее измеренные данные, не учитывает специфики эксперимента и решает математическую задачу восстановления без учета физических особенностей полученных входных данных.
Отсутствие учета особенностей измерительной схемы является одной из причин возникновения искажений в полученных изображениях --- так называемых артефактов восстановления.
Для борьбы с артефактами восстановления необходимо вносить в алгоритмы поправки, учитывающие особенности эксперимента, или разрабатывать принципиально новые схемы восстановления данных томографических изерений.
Это может быть эвристическая регуляризация промежуточного результата в процессе итеративного восстановления, модификация постановки оптимизационной задачи, учет физических явлений, проявляющихся только в конкретной экспериментальной схеме и др.

Отсутствие учета специфики проводимых исследований --- не единственная причина возникновения артефактов. 
Неточности, порождающие ошибки обработки, появляются на каждом этапе формирования картны измерений: аппаратная часть, восстановление и интерпретация полученных результатов.
Ошибки аппаратной части появляются в физической калибровке измерительной аппаратуры, калибровке геометрического расположения элементов в измерительной схеме, а так же при формировании входного сигнала для алгоритмов восстановления. 
К первым можно отнести ошибку в величине суммарной интенсивности излучения источника, в результате чего восстановленная картина может получиться ``пересвеченной''.
Примером второй причины может служить экспериментальная схема, в которой ось вращения объекта смещена относительно оси ``источник-детектор''.
Наконец, к погрешностям, вносимым аппаратурой, можно отнести шумы матрицы детектора, или наоборот, пересвечивание пикселей детектора.
На этапе алгоритмов восстановления могут быть внесены ошбики, связанные с вычислительной точностью программного обеспечения, сходимостью оптимизационных процедур (локальный минимум, недостижение минимума), слишком агрессивной регуляризацией. 
Последняя возможная причина состоит из ошибок визуализации трехмерных восстановленных картин и неправильной их интерпретации.
\end{comment}

%{\aim} ~данной работы являются разработка, теоретическое обоснование и тестирование  новых алгоритмов компьютерной томографии для восстановления измерений при зондировании рентгеновским монохроматическим и полихроматическим излучением. В частности, 

{\aim} ~данной работы являются разработка, обоснование и тестирование новых методов реконструкции в задаче компьютерной томографии, их эффективная численная реализация, а также создание метода интерпретации результатов томографического эксперимента в полихроматической моде, в частности

\begin{enumerate}
\item разработка и реализация асимптотически быстрых алгебраических методов реконструкции в задаче компьютерной томографии
\item разработка новых математических моделей для решения задач восстановления при наличии неисключаемых причин ошибок
\item разработка и численное тестирование алгоритмов восстановления в контексте построенных математических моделей
\end{enumerate}

Для достижения поставленных целей были решены следующие {\tasks}:
\begin{enumerate}
  \item построен асимптотически быстрый алгебраический метод реконструкции, основанный на применении быстрого преобразования Хафа.
  \item доказана сходимость построеного алгебраического метода реконструкции, за счет полученного математического выражения градиента быстрого преобразования Хафа.
  \item построен алгоритм реконструкции для объектов, содержащих сильнопоглощающие включения.
  \item построен алгоритм реконструкции, учитывающий покомпонентное ослабление полихроматического спектра.
\end{enumerate}

{\novelty}
\begin{enumerate}
  \item Впервые для реконструкции томографических измерений было применено быстрое преобразование Хафа.
  \item Впервые получено выражение для производной быстрого преобразования Хафа, а так же алгоритм его эффективного вычисления.
  \item Построен алгоритм реконструкции, учитывающий вклад сильно поглощающих включений с помощью оригинальной модели ограничений-неравенств.
  \item Предложена схема обработки данных полихроматического зондирования, при которой восстанавливаются реальные физические концентрации элементов.
\end{enumerate}

{\influence} ~Результаты, полученные в диссертационной работе, используются для обработки данных лабораторных исследований. Построенные алгоритмы лягут в основу программного обеспечения новых моделей промышленных томографов.

Полученное в работе выражение для градиента быстрого преобразования Хафа имеет общетеоретическое значение и уже применяется в области машинного обучения для обратного распространения ошибки в нейронных сетях глубокого обучения через слой БПХ.

{\methods}
Для решения задач реконструкции томографических измерений используются методы теории условной и безусловной оптимизации: градиентные методы оптимизации, квадратичное программирование, регуляризация.
Для ускорения итерации алгебраического метода используются алгоритмы обработки изображений в виде быстрого преобразования Хафа.


{\defpositions}
\begin{enumerate}
  \item Предложен эффективный вычислительный метод решения задачи томографической реконструкции FHT-SIRT, основанный на алгебраическом подходе, который позволяет снизить асимптотическую оценку сложности вычисления итерации с $O(n^3)$ до $O(n^2~\log n)$, что подтвержается численными экспериментами и замерами времени работы программной реализации алгоритма.
  \item Проведено математическое обоснование сходимости предложенного метода.
  \item Предложен метод реконструкции на основе квадратичного программирования, который позволяет уменьшить артефакты на восстановленном изображении, вызванные наличием сильно поглощающих областей в зондируемом объекте.
  \item Предложен алгебраический метод реконструкции для случая полихроматического зондирования, который решает оптимизационную задачу реконструкции относительно линейной комбинации концентраций с ограничениями-неравенствами на их область значений.
\end{enumerate}


{\reliability} полученных результатов обеспечивается модельными экспериментами и численными симуляциями, а так же экспериментами с восстановлением реально измеренных в лабораторных условиях образцов.\ Результаты находятся в соответствии с результатами, полученными другими авторами.


{\probation}
Основные результаты работы докладывались~на: конферециях 
35-я конференция молодых ученых и специалистов «Информационные технологии и системы» (2012, 19 - 25 августа, Петрозаводск, Россия),
11th Biennal Conference on High Resolution X-Ray Diffraction and Imaging (XTOP 2012, St. Petersburg, Russia), 
29th European Conference on Modelling and Simulation (ECMS 2015, Albena, Bulgaria),
Eighth International Conference on Machine Vision (ICMV 2015, Barcelona, Spain),
на общефизическом семинаре ИПТМ РАН (октябрь 2016).

{\contribution} Все результаты диссертации, вынесенные на защиту, получены автором самостоятельно.
Автором самостоятельно реализованы методы восстановления FHT-SIRT из первой главы, барьерных функций из второй, метод взвешанных невязок из третьей, проведены численные экспериметны по обработке реальных и модельных данных.
Постановка задач и обсуждение результатов проводились совместно с научным руководителем.
Генерация модельных данных для экспериментов в полихроматике проводилась аспирантом факультета КН НИУ ВШЭ Ингачевой А.~С. 
Программная имплементация метода мягких ограничений, использованная для сравнения с методом барьерных функций во второй главе, принадлежит Соколову В.~В.
Измерения для экспериментов по восстановлению зуба со свинцовым включением производились на лабоработном источнике ИК РАН в лаборатории рефлектометрии и малоуглового рассеяния.
Многие аспекты исследований в разное время обсуждались с Чукалиной М.~В., Николаевым Д.~П., Бузмаковым А.~В., Ингачевой А.~С., Соколовым В.~В.


\ifthenelse{\equal{\thebibliosel}{0}}{% Встроенная реализация с загрузкой файла через движок bibtex8
    \publications\ Основные результаты по теме диссертации изложены в 11 печатных изданиях, 
    3 из которых изданы в журналах, рекомендованных ВАК, 
    6 "--- в тезисах докладов.%
}{% Реализация пакетом biblatex через движок biber
%Сделана отдельная секция, чтобы не отображались в списке цитированных материалов
  \begin{refsection}%
    \printbibliography[heading=countauthorvak, env=countauthorvak, keyword=biblioauthorvak, section=1]%
    \printbibliography[heading=countauthornotvak, env=countauthornotvak, keyword=biblioauthornotvak, section=1]%
    \printbibliography[heading=countauthorconf, env=countauthorconf, keyword=biblioauthorconf, section=1]%
    \printbibliography[heading=countauthor, env=countauthor, keyword=biblioauthor, section=1]%

    \publications\ Основные результаты по теме диссертации изложены в \arabic{citeauthor} печатных изданиях \nocite{PruBuzNik13, Prun2013Crys, Vestnik2016, Prun2013AutomAndRemCont, buz_jac_2015, chukalina2014xray}, 
    \arabic{citeauthorvak} из которых изданы в журналах, рекомендованных ВАК %\nocite{PruBuzNik13, Prun2013Crys, Vestnik2016}, 
    \arabic{citeauthorconf} "--- в тезисах докладов \nocite{itas2015Prun,itas2015Ingacheva,ecms2015Chukalina, icmv2015Chukalina, embc2013Buzmakov, nikolaevfast}.
  \end{refsection}
}