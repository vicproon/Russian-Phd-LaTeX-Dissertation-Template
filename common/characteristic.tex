{\actuality} Для успешного решения многих прикладных задач необходимой является возможность подробного исследования внутренней структуры объектов без их физического изменения или разрушения.
Речь идет о таких областях применения как медицина, промышленность, биология, геология и др. 
Методы, позволяющие осуществлять такие исследования, назыавются томографическими.


\begin{comment}
Современное развитие вычислительной техники, повышение ее доступности, появление новых инструментариев обработки информации, развитие алгоритмов и компьютерных наук в целом, делают возможным применение алгоритмов обработки изображений для анализа результатов томографических измерений.
\end{comment}
В данной работе рассматриваются методы обработки измерений в методе рентгеновской томографии.
Исследуемый объект зондируется рентгеновским излучением под разными углами, называемыми проекционными.
Данные томографической проекции, т.е. измеренное ослабление рентгеновского излучения, позволяют восстановить внутренние характеристики объекта, а именно пространственное распределение коэффициента ослабления рентгеновского излучения.

Центральную роль в подобных измерениях играет томограф - программно-аппаратный комплекс, в котором совмещены как измерительная часть, так и математическая обработка полученных данных.
Аппаратная часть состоит из источника, держателя образца, оптики и детектора, регистрирующего прошедшее излучение. 
При этом возможны различные конфигурации эксперимента. 
Могут меняться форма зондирующего пучка, его спектр, спектр чувствительности и тип детектора, взаимное расположение детектора, источника и держателя образца.
Зачастую, например, в медицинских измерениях, критическую роль играет еще и доза облучения, поглащаемая объектом.

``Сырые данные'', полученные непосредственно после измерений не пригодны для их визуального анализа экспертом.
Для того чтобы получить из измеренных данных, так же называемых синограммой, искомые пространственные распределения характеристик объектов, необходимо произвести дополнительную обработку, называемую методами восстановления компьютерной томографии.
Программную часть томографа составляют алгоритмы восстановления томографических измерений, а так же программы управления механическими частями томографа, детектором и излучателем.
В данной работе рассматриваются проблемы, возникающие именно в алгоритмах восстановления, т.е. в программной части томографа.

\aim данной работы является создание и теоретическое обоснование новых алгоритмов восстановления изображений в рентгеновских методах томографии при зондировании монохроматическим и полихроматическим излучением.

Для~достижения поставленной цели были решены следующие {\tasks}:
\begin{enumerate}
  \item построен асимптотически быстрый алгебраический метод реконструкции, основанный на применении быстрого преобразования Хафа.
  \item доказана сходимость построеного алгебраического метода реконструкции, за счет полученного математического выражения градиента быстрого преобразования Хафа.
  \item построен алгоритм реконструкции для объектов, содержащих сильнопоглощающие включения.
  \item построен алгоритм реконструкции, учитывающий покомпонентное ослабление полихроматического спектра.
\end{enumerate}

{\novelty}
\begin{enumerate}
  \item Впервые для реконструкции томографических измерений было применено быстрое преобразование Хафа.
  \item Впервые получено аналитическое выражение для градиента быстрого преобразования Хафа, а так же алгоритм его вычисления через само преобразование.
  \item Построена оригинальная модель формирования томографических измерений для объектов с сильнопоглащающими включениями.
  \item Предложена схема обработки данных полихроматического зондирования, при которой восстанавливаются реальные физические концентрации элементов.
\end{enumerate}
{\influence} Результаты, полученные в диссертационной работе, используются для обработки лабораторных данных. Построенные алгоритмы лягут в основу программного обеспечения новых моделей промышленных томографов.

{\defpositions}
\begin{enumerate}
  \item Снижена асимтотическая оценка сложность вычисления итерации алгебраического метода восстановления компьютерной томографии с $O(n^3)$ до $O(n^2~\log n)$
  \item Доказана сходимость построеного алгебраического метода реконструкции. 
  \item Третье положение
  \item Четвертое положение
\end{enumerate}


{\reliability} полученных результатов обеспечивается численными симуляциями с модельными данными, а так же экспериментами с восстановлением реально измеренных образцов.\ Результаты находятся в соответствии с результатами, полученными другими авторами.


{\probation}
Основные результаты работы докладывались~на: конферециях 
11th Biennal Conference on High Resolution X-Ray Diffraction and Imaging (XTOP 2012, St. Petersburg, Russia), 
29th European Conference on Modelling and Simulation (ECMS 2015, Albena, Bulgaria),
Eighth International Conference on Machine Vision (ICMV 2015, Barcelona, Spain),
на общефизическом семинаре ИПТМ РАН (октябрь 2016).

{\contribution} Автор принимал активное участие \ldots


\ifthenelse{\equal{\thebibliosel}{0}}{% Встроенная реализация с загрузкой файла через движок bibtex8
    \publications\ Основные результаты по теме диссертации изложены в 11 печатных изданиях, 
    4 из которых изданы в журналах, рекомендованных ВАК, 
    6 "--- в тезисах докладов.%
}{% Реализация пакетом biblatex через движок biber
%Сделана отдельная секция, чтобы не отображались в списке цитированных материалов
    \begin{refsection}%
        \printbibliography[heading=countauthornotvak, env=countauthornotvak, keyword=biblioauthornotvak, section=1]%
        \printbibliography[heading=countauthorvak, env=countauthorvak, keyword=biblioauthorvak, section=1]%
        \printbibliography[heading=countauthorconf, env=countauthorconf, keyword=biblioauthorconf, section=1]%
        \printbibliography[heading=countauthor, env=countauthor, keyword=biblioauthor, section=1]%
        \publications\ Основные результаты по теме диссертации изложены в \arabic{citeauthor} печатных изданиях\nocite{PruBuzNik13, Prun2013AutomAndRemCont, Prun2013Crys, Vestnik2016, Buzmakov:zc5004}, 
        \arabic{citeauthorvak} из которых изданы в журналах, рекомендованных ВАК\nocite{PruBuzNik13, Prun2013AutomAndRemCont, Prun2013Crys, Vestnik2016}, 
        \arabic{citeauthorconf} "--- в тезисах докладов\nocite{itas2015Prun,itas2015Ingacheva,ecms2015Chukalina, icmv2015Chukalina, embc2013Buzmakov, nikolaevfast}.
    \end{refsection}
}